%% Generated by Sphinx.
\def\sphinxdocclass{jupyterBook}
\documentclass[letterpaper,10pt,english]{jupyterBook}
\ifdefined\pdfpxdimen
   \let\sphinxpxdimen\pdfpxdimen\else\newdimen\sphinxpxdimen
\fi \sphinxpxdimen=.75bp\relax
\ifdefined\pdfimageresolution
    \pdfimageresolution= \numexpr \dimexpr1in\relax/\sphinxpxdimen\relax
\fi
%% let collapsible pdf bookmarks panel have high depth per default
\PassOptionsToPackage{bookmarksdepth=5}{hyperref}
%% turn off hyperref patch of \index as sphinx.xdy xindy module takes care of
%% suitable \hyperpage mark-up, working around hyperref-xindy incompatibility
\PassOptionsToPackage{hyperindex=false}{hyperref}
%% memoir class requires extra handling
\makeatletter\@ifclassloaded{memoir}
{\ifdefined\memhyperindexfalse\memhyperindexfalse\fi}{}\makeatother

\PassOptionsToPackage{warn}{textcomp}

\catcode`^^^^00a0\active\protected\def^^^^00a0{\leavevmode\nobreak\ }
\usepackage{cmap}
\usepackage{fontspec}
\defaultfontfeatures[\rmfamily,\sffamily,\ttfamily]{}
\usepackage{amsmath,amssymb,amstext}
\usepackage{polyglossia}
\setmainlanguage{english}



\setmainfont{FreeSerif}[
  Extension      = .otf,
  UprightFont    = *,
  ItalicFont     = *Italic,
  BoldFont       = *Bold,
  BoldItalicFont = *BoldItalic
]
\setsansfont{FreeSans}[
  Extension      = .otf,
  UprightFont    = *,
  ItalicFont     = *Oblique,
  BoldFont       = *Bold,
  BoldItalicFont = *BoldOblique,
]
\setmonofont{FreeMono}[
  Extension      = .otf,
  UprightFont    = *,
  ItalicFont     = *Oblique,
  BoldFont       = *Bold,
  BoldItalicFont = *BoldOblique,
]



\usepackage[Bjarne]{fncychap}
\usepackage[,numfigreset=1,mathnumfig]{sphinx}

\fvset{fontsize=\small}
\usepackage{geometry}


% Include hyperref last.
\usepackage{hyperref}
% Fix anchor placement for figures with captions.
\usepackage{hypcap}% it must be loaded after hyperref.
% Set up styles of URL: it should be placed after hyperref.
\urlstyle{same}


\usepackage{sphinxmessages}



        % Start of preamble defined in sphinx-jupyterbook-latex %
         \usepackage[Latin,Greek]{ucharclasses}
        \usepackage{unicode-math}
        % fixing title of the toc
        \addto\captionsenglish{\renewcommand{\contentsname}{Contents}}
        \hypersetup{
            pdfencoding=auto,
            psdextra
        }
        % End of preamble defined in sphinx-jupyterbook-latex %
        

\title{Imaginery Numbers}
\date{Sep 21, 2023}
\release{}
\author{Dennis Ng}
\newcommand{\sphinxlogo}{\vbox{}}
\renewcommand{\releasename}{}
\makeindex
\begin{document}

\pagestyle{empty}
\sphinxmaketitle
\pagestyle{plain}
\sphinxtableofcontents
\pagestyle{normal}
\phantomsection\label{\detokenize{MacTwgssA0-intro::doc}}


\sphinxAtStartPar
Welcome to my little web site about imaginery number
Hopeful may generate PDF/latex/book … somehow.

\sphinxAtStartPar
This is based on \sphinxhref{https://jupyterbook.org/en/stable/intro.html}{Jupyter Book}.  Just like everything in here, everything is just a result of my hobby and learning.

\sphinxAtStartPar
The logo is from \sphinxhref{https://en.wikipedia.org/wiki/Imaginary\_number}{the Wikipedia page on Imaginery Number}; copyright are theirs.

\sphinxAtStartPar
And later “chapters” is just the sample pages from the book of Jupyter Book.  Keep them as it is my testing whether it is the setup issue (if even they cannot be presented) or it is on my page content.  Sorry, incomplete, draft, …

\begin{DUlineblock}{0em}
\item[] \sphinxstylestrong{\Large In case of doubt … it is just a hobby not for “production”}
\end{DUlineblock}

\sphinxAtStartPar
I am not sure anything is correct here.  As said just my hobby and hence if you find anything error, mistake, omission, … etc., please do alert me.

\sphinxAtStartPar
Check out the content pages here to see more.  Just do not not trust it is right.  Take them at best as a starting point of your investigation. If I can make you curious about the world, that is all what I aim for!!!
\begin{itemize}
\item {} 
\sphinxAtStartPar
{\hyperref[\detokenize{MacTwgssA1-think::doc}]{\sphinxcrossref{1. Thinking in another dimension}}}

\item {} 
\sphinxAtStartPar
{\hyperref[\detokenize{MacTwgssA2-zero::doc}]{\sphinxcrossref{2. Imagine about nothing}}}

\item {} 
\sphinxAtStartPar
{\hyperref[\detokenize{MacTwgssA3-imgNo::doc}]{\sphinxcrossref{3. How to imaginery something not possible to exist}}}

\item {} 
\sphinxAtStartPar
{\hyperref[\detokenize{MacTwgssA4-splitImgNo::doc}]{\sphinxcrossref{4. Imagine another imaginery number and split it out}}}

\item {} 
\sphinxAtStartPar
{\hyperref[\detokenize{MacTwgssAx-App::doc}]{\sphinxcrossref{Appendix \sphinxhyphen{} may be needed for reference}}}

\item {} 
\sphinxAtStartPar
{\hyperref[\detokenize{intro::doc}]{\sphinxcrossref{Welcome to your Jupyter Book}}}

\item {} 
\sphinxAtStartPar
{\hyperref[\detokenize{markdown::doc}]{\sphinxcrossref{Markdown Files}}}

\item {} 
\sphinxAtStartPar
{\hyperref[\detokenize{notebooks::doc}]{\sphinxcrossref{Content with notebooks}}}

\item {} 
\sphinxAtStartPar
{\hyperref[\detokenize{markdown-notebooks::doc}]{\sphinxcrossref{Notebooks with MyST Markdown}}}

\item {} 
\sphinxAtStartPar
{\hyperref[\detokenize{fAe::doc}]{\sphinxcrossref{testing jupyter\sphinxhyphen{}book failed using fAe examples}}}

\item {} 
\sphinxAtStartPar
{\hyperref[\detokenize{fFy::doc}]{\sphinxcrossref{testing jupyter\sphinxhyphen{}book ok using fFy examples}}}

\item {} 
\sphinxAtStartPar
{\hyperref[\detokenize{fnn::doc}]{\sphinxcrossref{testing jupyter\sphinxhyphen{}book ok using fnn examples}}}

\item {} 
\sphinxAtStartPar
{\hyperref[\detokenize{testAnimated-Sinc-and-FT-example::doc}]{\sphinxcrossref{testAnimated\sphinxhyphen{}Sinc\sphinxhyphen{}and\sphinxhyphen{}FT\sphinxhyphen{}example}}}

\item {} 
\sphinxAtStartPar
{\hyperref[\detokenize{testNewton-outdatewarning::doc}]{\sphinxcrossref{testNewton\sphinxhyphen{}outdatewarning}}}

\end{itemize}

\sphinxstepscope


\chapter{1. Thinking in another dimension}
\label{\detokenize{MacTwgssA1-think:thinking-in-another-dimension}}\label{\detokenize{MacTwgssA1-think::doc}}
\begin{sphinxVerbatim}[commandchars=\\\{\}]
Thinking in another dimension

... even it does not exist
... even if it is impossible to think about
\end{sphinxVerbatim}

\begin{sphinxVerbatim}[commandchars=\\\{\}]
My aim is not to \PYGZdq{}teach\PYGZdq{} maths 
    or purely introduce you to any numbering system
    
My objective is to 
    try to see a way to get one to think outside the box
    or to see another dimension even if it does not exist

To appreciate different dimension of life...
       And even if not agree ... 

If not, at least to wonder, 
       And to be curious ...
            And even if it is impossible to imagine these
                and frankly even you know deep down that dimension does not even exist!
\end{sphinxVerbatim}

\begin{sphinxVerbatim}[commandchars=\\\{\}]
There are many maths I think very striking when I encounter it the very first time

\PYGZhy{} Chicken and Rabbit in one cage with 4 heads and 6 legs how many ...
\PYGZhy{} what is negative number
\PYGZhy{} It is not that /sqrt\PYGZob{}2\PYGZcb{} is irrational or not, but the proof by contradiction
\PYGZhy{} imaginery number is real and there are at least 3 kinds of them

\PYGZhy{}\PYGZhy{}\PYGZhy{}\PYGZgt{} real number\PYGZsq{}s infinity is larger than integer number\PYGZsq{}s infinity
        (technical terms is real number is uncountable)

\PYGZhy{}\PYGZhy{}\PYGZhy{}\PYGZgt{} any formal mathematical system which is better than arthimetic will 
        have statement in it that cannot be proven or disped in that system
        
        (Or Should we say Mathematical Systems Always Contain Unprovable Truths)
        And even worst (or brillant?) someone proved this theorem of unprovable truth!
    
\PYGZhy{}\PYGZhy{}\PYGZhy{}\PYGZgt{} in real world, triangle is either \PYGZgt{}=\PYGZlt{} 180 degree, which one then?
        only one is the truth and is real
        then why bother and what is the point of having 3 geometries
        and even worst ... find all of them are useful!!!
    
\PYGZhy{}\PYGZhy{}\PYGZgt{} maths is useful and relevant 
        But most suprising at all, is why maths is useful at all?
        Why a human endeavour can have relevance to the world?
        And what is the limit of this approach to life?

OK too much BS, let use start with a simple 

\PYGZhy{}\PYGZhy{}\PYGZgt{} projective geometry first
\end{sphinxVerbatim}

\begin{sphinxVerbatim}[commandchars=\\\{\}]
Imagine you have a circle of say radius 1 that
    sit on a line of infinite length say the real number line
    
From the \PYGZdq{}north pole\PYGZdq{} of that circle you draw a line towards the line
    then move left or right and map each point of the circle to the line
    
Can you see that each point of the circle can project to each point to the line
    (techncailly not the +/\PYGZhy{} infinite as you only have one north pole la
     can +/\PYGZhy{} infinite meant? let us not go that far.  But even a bit short ...
     do you find it amazing?

How can it be?
The circle has a circumference of 2pi
The length of the line is ... infinity 
How can you map finite to infinitity

Can you really see the world in a sand?

Or even worst as in my most dislike Fan Yin (or no feeling Indian) 
    the world is all but a reflection of the pearl 因佗羅網
    Just watch one water dew on the leave you can see the whole world in it
\end{sphinxVerbatim}

\sphinxAtStartPar
Anyway … next let us look at some number system as I promise to present on at least one!

\sphinxstepscope


\chapter{2. Imagine about nothing}
\label{\detokenize{MacTwgssA2-zero:imagine-about-nothing}}\label{\detokenize{MacTwgssA2-zero::doc}}
\begin{sphinxVerbatim}[commandchars=\\\{\}]
 Leopold Kronecker, who once wrote that 
 
 \PYGZhy{}\PYGZhy{}\PYGZgt{} \PYGZdq{}God made the integers; all else is the work of man.\PYGZdq{}
 
     \PYGZhy{} a key and very dominant mathematicans who object irrational number, pi, set theory, ... etc. 
     \PYGZhy{} It was finally proved that pi is transcendetal number ...
     \PYGZhy{}  but as he said what is the point, they do not \PYGZdq{}exist\PYGZdq{} ! Illusion!!!
     \PYGZhy{} (In a way he is not alone, 
             a lot of mathmaticians are also not very uncomforable with negative number ...)

 \PYGZhy{}\PYGZhy{}\PYGZgt{} Let us do zero first and then imaginery number
\end{sphinxVerbatim}

\begin{sphinxuseclass}{cell}\begin{sphinxVerbatimInput}

\begin{sphinxuseclass}{cell_input}
\begin{sphinxVerbatim}[commandchars=\\\{\}]
\PYG{k+kn}{import} \PYG{n+nn}{os}

\PYG{n+nb}{print}\PYG{p}{(}\PYG{n}{os}\PYG{o}{.}\PYG{n}{getcwd}\PYG{p}{(}\PYG{p}{)}\PYG{p}{)}
\end{sphinxVerbatim}

\end{sphinxuseclass}\end{sphinxVerbatimInput}
\begin{sphinxVerbatimOutput}

\begin{sphinxuseclass}{cell_output}
\begin{sphinxVerbatim}[commandchars=\\\{\}]
/Users/ngcchk/Documents/GitHub/gpd2\PYGZhy{}win\PYGZhy{}unity1/ipadred\PYGZhy{}rain/imgno\PYGZus{}book1/imgnobk1
\end{sphinxVerbatim}

\end{sphinxuseclass}\end{sphinxVerbatimOutput}

\end{sphinxuseclass}
\begin{sphinxuseclass}{cell}\begin{sphinxVerbatimInput}

\begin{sphinxuseclass}{cell_input}
\begin{sphinxVerbatim}[commandchars=\\\{\}]
\PYG{k+kn}{import} \PYG{n+nn}{lib}\PYG{n+nn}{.}\PYG{n+nn}{main}\PYG{n+nn}{.}\PYG{n+nn}{a0\PYGZus{}babylon\PYGZus{}pos\PYGZus{}0}
\end{sphinxVerbatim}

\end{sphinxuseclass}\end{sphinxVerbatimInput}
\begin{sphinxVerbatimOutput}

\begin{sphinxuseclass}{cell_output}
\begin{sphinxVerbatim}[commandchars=\\\{\}]
not in main of a0
\end{sphinxVerbatim}

\end{sphinxuseclass}\end{sphinxVerbatimOutput}

\end{sphinxuseclass}
\begin{sphinxuseclass}{cell}\begin{sphinxVerbatimInput}

\begin{sphinxuseclass}{cell_input}
\begin{sphinxVerbatim}[commandchars=\\\{\}]
\PYG{n}{lib}\PYG{o}{.}\PYG{n}{main}\PYG{o}{.}\PYG{n}{a0\PYGZus{}babylon\PYGZus{}pos\PYGZus{}0}\PYG{o}{.}\PYG{n}{display\PYGZus{}img}\PYG{p}{(}\PYG{l+m+mi}{1}\PYG{p}{)}
\end{sphinxVerbatim}

\end{sphinxuseclass}\end{sphinxVerbatimInput}
\begin{sphinxVerbatimOutput}

\begin{sphinxuseclass}{cell_output}
\noindent\sphinxincludegraphics{{471baaa0194945d1ab434b498e8531a1c6ea386bbd7c0c99339311f2abe37783}.png}

\end{sphinxuseclass}\end{sphinxVerbatimOutput}

\end{sphinxuseclass}
\begin{sphinxVerbatim}[commandchars=\\\{\}]
        note 
        a) there are only 2 symbols 1 and 10
        b) every number is actually by placing and counting number 1 and number 10 out; FULLY
        c) reaching 59 then what ... 60 is a problem let us skip it first ;\PYGZhy{}P
\end{sphinxVerbatim}

\begin{sphinxuseclass}{cell}\begin{sphinxVerbatimInput}

\begin{sphinxuseclass}{cell_input}
\begin{sphinxVerbatim}[commandchars=\\\{\}]
\PYG{n}{lib}\PYG{o}{.}\PYG{n}{main}\PYG{o}{.}\PYG{n}{a0\PYGZus{}babylon\PYGZus{}pos\PYGZus{}0}\PYG{o}{.}\PYG{n}{display\PYGZus{}img}\PYG{p}{(}\PYG{l+m+mi}{2}\PYG{p}{)}
\end{sphinxVerbatim}

\end{sphinxuseclass}\end{sphinxVerbatimInput}
\begin{sphinxVerbatimOutput}

\begin{sphinxuseclass}{cell_output}
\noindent\sphinxincludegraphics{{39571f3cb2beee813dda02e06fe31aaa3beda6e9523d43a835a0beb6c571cc13}.png}

\end{sphinxuseclass}\end{sphinxVerbatimOutput}

\end{sphinxuseclass}
\begin{sphinxVerbatim}[commandchars=\\\{\}]
        d) Babylon use 60 as based and hence 1 is 60 * 60 *...
        
            e.g. 1 here is 60\PYGZca{}3 ...
        
        e) A number\PYGZsq{}s value is based on its position
        
            The 1 above is 1 * 60 * 60 * 60 because its postion after you read in the hole number
        
            In fact even worst you cannot tell what 1 meant until you have the whole number
                as it uses the Big Endian convention 
                i.e. until you reach the end of a number you do not know what is 1
                It can be 1, 60, 60*60, 60*60*... you do not know
                (Cf the little endian for this number 40, 46, 57, 1
                 and after reading 40 you know it is just 40, 
                 and after reading 46 as \PYGZdq{}nd number you know it is 46 * 60 ...
                 you do not need to wait for the whole number
                 or fight the egg head war of big ednian vs little endian )
                 
            The Roman does not use postional system and hence has no such issue
                The only trick to remember if you see a small number earlier than a larger number deduct it
                XL and LX where X is 10 and L is 50, what are the numbers ?
                You do not need to know the position of X or L is you know it is always 10 and 50
                In fact, there is no need of zero (except zero itself called nulla)
                
            Actually Hans\PYGZsq{}s does not really use a positional system 
                as 2,0001 is really 2 thousands and one not 2001
                still have zero to sound better
                 
        f) For the issue 1574640 is 15,7 or 1,57 ... Babylon use gap
        
            i.e. 1 gap 57 gap 46 gap 40
            
        g) How about 60, 600, ... 
        
            seems to rely upon common sense?
        
        h) But what about 601 6001 ... Big problem
        
           The problem is that there is no zero and they use a positional system!!!
\end{sphinxVerbatim}

\begin{sphinxuseclass}{cell}\begin{sphinxVerbatimInput}

\begin{sphinxuseclass}{cell_input}
\begin{sphinxVerbatim}[commandchars=\\\{\}]
\PYG{n}{lib}\PYG{o}{.}\PYG{n}{main}\PYG{o}{.}\PYG{n}{a0\PYGZus{}babylon\PYGZus{}pos\PYGZus{}0}\PYG{o}{.}\PYG{n}{display\PYGZus{}img}\PYG{p}{(}\PYG{l+m+mi}{3}\PYG{p}{)}
\end{sphinxVerbatim}

\end{sphinxuseclass}\end{sphinxVerbatimInput}
\begin{sphinxVerbatimOutput}

\begin{sphinxuseclass}{cell_output}
\noindent\sphinxincludegraphics{{ee9ec6e7b82ed8a727bce0a03d5c809f3e8773a8ced5fc0012603ae2cbd4a6e7}.png}

\end{sphinxuseclass}\end{sphinxVerbatimOutput}

\end{sphinxuseclass}
\begin{sphinxVerbatim}[commandchars=\\\{\}]
\PYGZdq{}a cuneiform tablet (actually AO 17264 in the Louvre collection in Paris)\PYGZdq{}

we have

    2 ? 27    ??         6      ?          9  

     gap,   large gap,    medium size gap 
     
What is it?

    2,27 square is 6 \PYGZlt{}0\PYGZgt{} 9 ...

    2\PYGZlt{}gap\PYGZgt{}27  \PYGZlt{}very big gap\PYGZgt{}   6\PYGZlt{}larger gap\PYGZgt{}9
    
    
    
This is intrepreted as 

2*60+27 = 147 * 147 = 21,609 or 6*60*60+9 

Thanks you very much for the attention!!!!!
\end{sphinxVerbatim}

\begin{sphinxVerbatim}[commandchars=\\\{\}]
Babylon consider zero as nothingness, and does not exist
But they need at least in their positional based number system

Even if zero does not exist
but one may have to accept its usefulness
and use it ... at least that is what the Babylonian did
\end{sphinxVerbatim}

\begin{sphinxVerbatim}[commandchars=\\\{\}]
You may think it does not exist, but your system really need it wow!!!!

IT DOES NOT EXIST
IT IS NOT REAL 
THERE IS NOTHING THERE!!!!

Still...
\end{sphinxVerbatim}

\begin{sphinxVerbatim}[commandchars=\\\{\}]
Many maths and possibly many religion and philosophy are like that!!! 

As said only integer really exist, all are ... (the guy is mad mathematicans btw)!
\end{sphinxVerbatim}

\sphinxstepscope


\chapter{3. How to imaginery something not possible to exist}
\label{\detokenize{MacTwgssA3-imgNo:how-to-imaginery-something-not-possible-to-exist}}\label{\detokenize{MacTwgssA3-imgNo::doc}}
\begin{sphinxVerbatim}[commandchars=\\\{\}]
actually the imaginery number is unusual, but not that unusual. 

Like negative number and zero it does not exist.  

But it exist somehow in some way and you know it, like the zero in Babylonian system.  

You need it, even though, well, it does not exist.
\end{sphinxVerbatim}

\begin{sphinxVerbatim}[commandchars=\\\{\}]
The major turning point is in the climbing of grease pole of being a maths professor!

In the video I provided (see https://www.youtube.com/watch?v=cUzklzVXJwo), one of the challenge
at that time is the equation of 

\PYGZdl{}\PYGZdl{}
    \PYGZob{}x\PYGZca{}3\PYGZcb{} = 15x + 4
\PYGZdl{}\PYGZdl{}

Now this equation can be solved unlike \PYGZdl{}x\PYGZca{}\PYGZob{}2\PYGZcb{} = \PYGZhy{}1\PYGZdl{} which we can say there is no solution.  
Simple! 4 is an answer!!! But then we have a problem, the general equation leads to this:

\PYGZdl{}\PYGZdl{}
    \PYGZbs{}sqrt[3]\PYGZob{}2 + \PYGZbs{}sqrt\PYGZob{}\PYGZhy{}121\PYGZcb{}\PYGZcb{} + \PYGZbs{}sqrt[3]\PYGZob{}2 \PYGZhy{} \PYGZbs{}sqrt\PYGZob{}\PYGZhy{}121\PYGZcb{}\PYGZcb{}
\PYGZdl{}\PYGZdl{}

or \PYGZob{}2+(\PYGZhy{}121)\PYGZca{}(1/2)\PYGZcb{}\PYGZca{}(1/3) + \PYGZob{}2\PYGZhy{}(\PYGZhy{}121)\PYGZca{}(1/2)\PYGZcb{}\PYGZca{}(1/3) 

in fact this complex expression is just 4, 

    as it turns out it that if you treat \PYGZbs{}sqrt[3]\PYGZob{}2 + \PYGZbs{}sqrt\PYGZob{}\PYGZhy{}121\PYGZcb{}\PYGZcb{}  as a + bi
    and \PYGZbs{}sqrt[3]\PYGZob{}2 \PYGZhy{} \PYGZbs{}sqrt\PYGZob{}\PYGZhy{}121\PYGZcb{}\PYGZcb{} as a \PYGZhy{} bi then one can find out that 

    \PYGZob{}2+(\PYGZhy{}121)\PYGZca{}(1/2)\PYGZcb{}\PYGZca{}(1/3) = 2+i and
    \PYGZob{}2\PYGZhy{}(\PYGZhy{}121)\PYGZca{}(1/2)\PYGZcb{}\PYGZca{}(1/3) = 2\PYGZhy{}i and together it is just 

    4

in fact once we know 4 is a factor you can easily found out all the REAL root of this equation:

i.e. 4, 
    \PYGZdl{}\PYGZhy{}2\PYGZhy{}\PYGZbs{}sqrt\PYGZob{}3\PYGZcb{}\PYGZdl{} and 
    \PYGZdl{}\PYGZhy{}2+\PYGZbs{}sqrt\PYGZob{}3\PYGZcb{}\PYGZdl{} 
    
and the equation would be 

\PYGZdl{}\PYGZdl{}
    (x\PYGZhy{}4)(x\PYGZhy{}(\PYGZhy{}2\PYGZhy{}\PYGZbs{}sqrt\PYGZob{}3\PYGZcb{}))(x\PYGZhy{}(\PYGZhy{}2+\PYGZbs{}sqrt\PYGZob{}3\PYGZcb{}))
\PYGZdl{}\PYGZdl{}

It is only the intermediate step use imaginery number!  

(Just like later quantum mechanics, the intermediate step use imaginary number, 
    but one can only observe real number ... a major debate leading to Coppehegan Interpretation,
    or just shut up and calculate!)
\end{sphinxVerbatim}

\begin{sphinxVerbatim}[commandchars=\\\{\}]
If imaginary numner i.e i\PYGZca{}2 = \PYGZhy{}1,  it does not exist it seems.

But could it be like the i above and the 0 in babylon above, 
we can treat it as existed but not ultimately.  

In fact this concept also apply to the trial of Galialo!  
(Just he does not accept this kind of argument.  No tool argument.  No just calculate...)
\end{sphinxVerbatim}

\begin{sphinxVerbatim}[commandchars=\\\{\}]
Our mind is bounded as we too used to the magic of the arthimetic since ancient times:
    
    一與言為二,二與一為三。自此以往,巧歷不能得,而況其凡乎!
    
        One and Speech are two; two and one are three. 
        Going on from this (in our enumeration), the most skilful reckoner cannot reach
        (the end of the necessary numbers), and how much less can ordinary people do so!
        see https://ctext.org/zhuangzi/adjustment\PYGZhy{}of\PYGZhy{}controversies
    
    道生一,一生二,二生三,三生萬物 
    
        \PYGZdq{}The Dao produced One; One produced Two; Two produced Three; Three produced All things.\PYGZdq{}
            see https://ctext.org/dao\PYGZhy{}de\PYGZhy{}jing
        (Unlike zhuang zi, Laozi may also touch on older modular view of I Ching
         but not necessarily Zhou version; both he and Zhuang Zi likely of one dynasty before
         One has to go back much earlier to know there might be at least 5 sources ....)

    +/* can stay on natural number stated above ... may have prime/composite number
    
    0 may or may not natural number ...
    
    \PYGZhy{}\PYGZhy{}\PYGZhy{}\PYGZhy{}\PYGZhy{}\PYGZhy{} that is gap here as pointed out by the \PYGZdq{}mad\PYGZdq{} maths guy
    
    Once we start to think backward then we have a lot of number/problem/issue:
    
    \PYGZhy{}   reverse of +
    
        rotation of 180 degree or pi radian?
        reflection
        shift
        already unnatural (0 by Indian, \PYGZhy{}ve by Chinese)
        generate integer
    
    / reverse of *
    
        have rational number (by Egyptians) 
        (technically can generate infinite but seems not historical from this line of thinking?)
        (lots of strange number like finite decimal, dyadic (final binary), repeating decimal
            but one must be careful, the reason why it is that because we try to express say
            1/3 as a sum of 1/10+...
    
    /sqrt reverse of one particular type of * 
    
        irrational especially possibly the first example of \PYGZdl{}/sqrt\PYGZob{}2\PYGZcb{}\PYGZdl{}
            (proved by geometry then)         
        ... algeraic irrational and a lot of others like our imaginery number
    
    ratio and change
        pi (nature unknown and only in 19th century prove to transcental like e)
        ratio of circumference and radius (radian is 2pi because of 2 pi r)
        e as the growth rate or change in y is equal to y 
    
    complex number 
        discussed here through algbera
    
    many strange number system now, but unlike many other systems, 
        they ARE NUMBER SYSTEM e.g. you can +/\PYGZhy{}/*///\PYGZca{} etc.
    
    But we most use real number (and in fact can stay on it even if we use imaginery number)
    
    


Let us SEE THE MAGIC
\end{sphinxVerbatim}

\begin{sphinxVerbatim}[commandchars=\\\{\}]
Note

    1 * i             = i         (i\PYGZca{}1)
    1 * i * i         = \PYGZhy{}1        (i\PYGZca{}2)
    1 * i * i * i     = \PYGZhy{}i        (i\PYGZca{}3)
    1 * i * i * i * i = 1         (i\PYGZca{}5)
    
In fact one can see that every 4 operation of i comes back to 1

What is that operation
\end{sphinxVerbatim}

\begin{sphinxVerbatim}[commandchars=\\\{\}]
ROTATION! to be exact rotation of 90 degree or pi/2 radian

i is not on real number line and is NOT a real number no doubt, 
but could be it is a rotation of real number to another dimension

Things does not exist in the real world may not have consequences in the real world

Think

Outside the box

北冥有魚,其名為鯤。鯤之大,不知其幾千里也。
化而為鳥,其名為鵬。鵬之背,不知其幾千里也;
怒而飛,其翼若垂天之雲。是鳥也,海運則將徙於南冥

In the Northern Ocean there is a fish, the name of which is Kun 
    \PYGZhy{} I do not know how many li in size. 
It changes into a bird with the name of Peng, 
    the back of which is (also) \PYGZhy{} I do not know how many li in extent. 
When this bird rouses itself and flies, 
    its wings are like clouds all round the sky. 
    When the sea is moved (so as to bear it along), 
    it prepares to remove to the Southern Ocean. 
    The Southern Ocean is the Pool of Heaven.
    see https://ctext.org/zhuangzi/enjoyment\PYGZhy{}in\PYGZhy{}untroubled\PYGZhy{}ease
    
   
Once we accept another sky dimension we can no longer bounded by the sea
   
    The metamorphosis of 鯤(Kun)鵬(Peng), no longer a fish but a bird!!!
\end{sphinxVerbatim}

\begin{sphinxVerbatim}[commandchars=\\\{\}]
Once we open up i as another dimension 

In fact it has to be 90 degree from real number dimension, 

i and 1 could easily AND NATURALLY form a number with 2 dimensional

The other trick now is because it comes from rotation, 
    there is a lot of usefulness to it and it is now everywhere!
\end{sphinxVerbatim}

\begin{sphinxuseclass}{cell}\begin{sphinxVerbatimInput}

\begin{sphinxuseclass}{cell_input}
\begin{sphinxVerbatim}[commandchars=\\\{\}]
\PYG{o}{\PYGZpc{}}\PYG{k}{matplotlib} inline
\PYG{k+kn}{import} \PYG{n+nn}{lib}\PYG{n+nn}{.}\PYG{n+nn}{main}\PYG{n+nn}{.}\PYG{n+nn}{a1\PYGZus{}i\PYGZus{}mul\PYGZus{}plot}
\end{sphinxVerbatim}

\end{sphinxuseclass}\end{sphinxVerbatimInput}
\begin{sphinxVerbatimOutput}

\begin{sphinxuseclass}{cell_output}
\begin{sphinxVerbatim}[commandchars=\\\{\}]
lib.main.a1\PYGZus{}i\PYGZus{}mul\PYGZus{}plot
not via main
it is a rotation
and i always rotate 90 degree or pi/2 radians
\end{sphinxVerbatim}

\noindent\sphinxincludegraphics{{f63e9dabeacd2459c5b43c505f880985e3bac2a9972f27577770d543149abeff}.png}

\end{sphinxuseclass}\end{sphinxVerbatimOutput}

\end{sphinxuseclass}
\begin{sphinxuseclass}{cell}\begin{sphinxVerbatimInput}

\begin{sphinxuseclass}{cell_input}
\begin{sphinxVerbatim}[commandchars=\\\{\}]
\PYG{o}{\PYGZpc{}}\PYG{k}{matplotlib} inline
\PYG{n}{lib}\PYG{o}{.}\PYG{n}{main}\PYG{o}{.}\PYG{n}{a1\PYGZus{}i\PYGZus{}mul\PYGZus{}plot}\PYG{o}{.}\PYG{n}{some\PYGZus{}plot}\PYG{p}{(}\PYG{p}{)}
\end{sphinxVerbatim}

\end{sphinxuseclass}\end{sphinxVerbatimInput}
\begin{sphinxVerbatimOutput}

\begin{sphinxuseclass}{cell_output}
\begin{sphinxVerbatim}[commandchars=\\\{\}]
\PYGZhy{}\PYGZhy{} more example\PYGZhy{}\PYGZhy{}
\end{sphinxVerbatim}

\noindent\sphinxincludegraphics{{9ca01360ece7cb0f6ae239c6c5012e88cf0789811fb1cd49ba3b5777042089a7}.png}

\noindent\sphinxincludegraphics{{1d5ca392f2e0f54d998cd34b2d2870c98a5d4ac536eb8c8d970717b62ff8c8ba}.png}

\noindent\sphinxincludegraphics{{0fec07713bc70a91cfdfe4060246b271e12c72622281031e5fe466f8ae10ca6a}.png}

\noindent\sphinxincludegraphics{{2eb9c53133a9d2d01655f13ca2c5d6f787a8bed41fc77dc560ce5d5df7baf655}.png}

\noindent\sphinxincludegraphics{{253e81a4e934cbfa4f42f4c626fd661a9986be6bb6d6da961912efdb7d3cc50a}.png}

\noindent\sphinxincludegraphics{{f63e9dabeacd2459c5b43c505f880985e3bac2a9972f27577770d543149abeff}.png}

\noindent\sphinxincludegraphics{{79f79dc8d0bb4c3553822d24147d965a264fddf2af996a19d335228e07073390}.png}

\noindent\sphinxincludegraphics{{8ff538469fc018881da1b25af26f1b55ce9889795eb6c7f6b84fcd2dd99e589c}.png}

\noindent\sphinxincludegraphics{{9ca01360ece7cb0f6ae239c6c5012e88cf0789811fb1cd49ba3b5777042089a7}.png}

\end{sphinxuseclass}\end{sphinxVerbatimOutput}

\end{sphinxuseclass}
\begin{sphinxuseclass}{cell}\begin{sphinxVerbatimInput}

\begin{sphinxuseclass}{cell_input}
\begin{sphinxVerbatim}[commandchars=\\\{\}]
\PYG{c+c1}{\PYGZsh{}import lib.main.e5\PYGZus{}i\PYGZus{}anim\PYGZus{}hyperplot }
\PYG{n+nb}{print} \PYG{p}{(}\PYG{l+s+s2}{\PYGZdq{}}\PYG{l+s+s2}{c3; red start and blue end}\PYG{l+s+s2}{\PYGZdq{}}\PYG{p}{)}
\PYG{k+kn}{import} \PYG{n+nn}{lib}\PYG{n+nn}{.}\PYG{n+nn}{main}\PYG{n+nn}{.}\PYG{n+nn}{c3\PYGZus{}i\PYGZus{}anim\PYGZus{}circle\PYGZus{}rotate\PYGZus{}plot}
\PYG{k+kn}{import} \PYG{n+nn}{numpy} \PYG{k}{as} \PYG{n+nn}{np}
\PYG{n}{lib}\PYG{o}{.}\PYG{n}{main}\PYG{o}{.}\PYG{n}{c3\PYGZus{}i\PYGZus{}anim\PYGZus{}circle\PYGZus{}rotate\PYGZus{}plot}\PYG{o}{.}\PYG{n}{plot\PYGZus{}i\PYGZus{}circle}\PYG{p}{(}\PYG{n}{np}\PYG{o}{.}\PYG{n}{pi}\PYG{o}{/}\PYG{l+m+mi}{4}\PYG{p}{)}
\end{sphinxVerbatim}

\end{sphinxuseclass}\end{sphinxVerbatimInput}
\begin{sphinxVerbatimOutput}

\begin{sphinxuseclass}{cell_output}
\begin{sphinxVerbatim}[commandchars=\\\{\}]
c3; red start and blue end
lib.main.e5\PYGZus{}i\PYGZus{}anim\PYGZus{}hyperplot
not via e5 main
/Users/ngcchk/Documents/GitHub/gpd2\PYGZhy{}win\PYGZhy{}unity1/ipadred\PYGZhy{}rain/imgno\PYGZus{}book1/imgnobk1
not vai c3 main

index: 43 
theta4 at  7.253072972046234 
 a= 0.5653921912399589  b= 0.8248222051356752 
\end{sphinxVerbatim}

\noindent\sphinxincludegraphics{{01693f9a3c09b9199cd4106daee17d31a349cd4a69d0e086f41314fab0c64b6e}.png}

\end{sphinxuseclass}\end{sphinxVerbatimOutput}

\end{sphinxuseclass}
\begin{sphinxVerbatim}[commandchars=\\\{\}]
and if you change ø you can see 

\PYGZhy{} a one dimensin being (on x) a peridic change (like as real number being we see i,\PYGZhy{}1,\PYGZhy{}i,1 ...)
\PYGZhy{} or for 2 dimension being (on i and x), you can see a wave!!!
\end{sphinxVerbatim}

\begin{sphinxVerbatim}[commandchars=\\\{\}]
Any number on this 2 dimension plane can work like any other number
It can be added, subtracted, multiplicaion and divide.  (This is different from vector.)

Not just that, given we have a multiplication, and we know there is rotation involve, 
one can seperate any complex number into another 2 dimension (not like x and y), 
but its multiplcaiton character and its rotation character.

Looking at real number
c = 2 * 3 
c will be double of 3, 
or if we * c it will be 6 *
hence we know * z would have one dimension of multiplication of its magnitude, so ...

z = magitudeof z * (z / magitude of z) 
or |Z| * Z/|Z| 

i.e. we know that 
*z will mean using |Z| to multiply and Z/|Z| to rotate

In fact we call this the polar form i.e. z = |Z| * f(ø) or A** f(ø)

\PYGZhy{}\PYGZhy{}\PYGZgt{} and in the diagram we can even represent it as

z = A*(cosø+isinø)

BTW to calculate A one easy way because complex number is 90 degree rotation,
    it follows Euclidian Geometry or
    z = a + bi then |z| = /sqrt\PYGZob{}a\PYGZca{}2 + b\PYGZca{}2\PYGZcb{} 
      = (a + bi)(a \PYGZhy{} bi) where a\PYGZhy{}bi is the conjugate of z
    we will find this useful when the rotation is not Euclidian ...
\end{sphinxVerbatim}

\begin{sphinxuseclass}{cell}\begin{sphinxVerbatimInput}

\begin{sphinxuseclass}{cell_input}
\begin{sphinxVerbatim}[commandchars=\\\{\}]
\PYG{n+nb}{print} \PYG{p}{(}\PYG{l+s+s2}{\PYGZdq{}}\PYG{l+s+s2}{another example \PYGZhy{} c3; red start and blue end}\PYG{l+s+s2}{\PYGZdq{}}\PYG{p}{)}
\PYG{k+kn}{import} \PYG{n+nn}{lib}\PYG{n+nn}{.}\PYG{n+nn}{main}\PYG{n+nn}{.}\PYG{n+nn}{c3\PYGZus{}i\PYGZus{}anim\PYGZus{}circle\PYGZus{}rotate\PYGZus{}plot}
\PYG{k+kn}{import} \PYG{n+nn}{numpy} \PYG{k}{as} \PYG{n+nn}{np}
\PYG{c+c1}{\PYGZsh{}lib.main.c3\PYGZus{}i\PYGZus{}anim\PYGZus{}circle\PYGZus{}rotate\PYGZus{}plot.do\PYGZus{}anim()}
\PYG{c+c1}{\PYGZsh{} need to run in terminal for the moment???}
\end{sphinxVerbatim}

\end{sphinxuseclass}\end{sphinxVerbatimInput}
\begin{sphinxVerbatimOutput}

\begin{sphinxuseclass}{cell_output}
\begin{sphinxVerbatim}[commandchars=\\\{\}]
another example \PYGZhy{} c3; red start and blue end
\end{sphinxVerbatim}

\end{sphinxuseclass}\end{sphinxVerbatimOutput}

\end{sphinxuseclass}
\begin{sphinxVerbatim}[commandchars=\\\{\}]
z = A*(cosø+isinø)

This does not help us very much as a forumla goes

If we rotate z1 and the z2

z1*z2 invoving cos and sin and is very painful

\PYGZhy{}\PYGZhy{}\PYGZhy{}\PYGZhy{} 

However, euler has studied and found out that

e\PYGZca{}iø = cosø + isinø  \PYGZlt{}\PYGZhy{}\PYGZhy{} the euler equation

substitue it one can get the rotation part is actually e\PYGZca{}iø

or z = Ae\PYGZca{}iø

For example z1*z2 = A1*A2(e\PYGZca{}iø1)(e\PYGZca{}iø2) = A1*A2*e\PYGZca{}i(ø1+ø2) 
        
             \PYGZlt{}\PYGZhy{}\PYGZhy{} NO cosine and sine but multiplication, addition and power
             (another thankful one is (cosø + isinø)\PYGZca{}n = (cosnø + isinnø) Moivre and Euler

This will help the next 100+ years of enginering, and maths and ... 
as rotation is everywhere and now additing of rotation is really just addition!
\end{sphinxVerbatim}

\begin{sphinxVerbatim}[commandchars=\\\{\}]
As beauty if one substitue pi into that euler equation, 
we have what many call the most beautiful mathmathical equation :

e\PYGZca{}ipi = \PYGZhy{}1

and may we note:

\PYGZhy{}1 is arthmetic unusal thing
pi is geometry  unusal thing
i  is algebera  unusal thing
e  is calculus  unusal thing

They are from total different field of human endeavour ... 
and somehow these unusal all join up into this

EULER IDENTITY
\end{sphinxVerbatim}

\begin{sphinxVerbatim}[commandchars=\\\{\}]
Unfortunately and may be fortunately someone observe that i nature is rotation 
and hence if one observe how i rotate, one can generate wave as seen above. 

And if one note that the change of this i is related back to itself, 
a guy actually create a quantum mechanic out of this imaginary number

change of wave over time = i * [...] * wave itself \PYGZlt{}\PYGZhy{}\PYGZhy{} basic Schrödinger equation

\PYGZhy{}\PYGZhy{}\PYGZhy{}\PYGZhy{} Part 1 finshed with some quantum mechanics puzzle



we can only observe real number
    The rotation is there but we still see the real number
    Like the one dimension man only see x not i in the rotation demo
Why one can use magnitude to observe as probability
Random is built\PYGZhy{}in and determine in the last instance and not before
And what happen to the The Schrödinger Cat then

Discrete and minimum energy, ... any other discrete

Uncertainty principles ...
\end{sphinxVerbatim}

\begin{sphinxVerbatim}[commandchars=\\\{\}]
BTW, real world offer more rotation ... 

    not only we have rotate 360 i,
    but we also rotate 720 degree to go back to itself 
    … and like the song \PYGZdq{}love always around us\PYGZdq{} in 4 wedding, it is all around us!
    may be in the future we talk about electron!
    
Also, you may note it is i, but actually someone has expanded into 4 dimension 
    and it is use in every computer game
    when you rotate a character you fit the transformation into this 4 dimension rotation
    ... 
\end{sphinxVerbatim}

\sphinxstepscope


\chapter{4. Imagine another imaginery number and split it out}
\label{\detokenize{MacTwgssA4-splitImgNo:imagine-another-imaginery-number-and-split-it-out}}\label{\detokenize{MacTwgssA4-splitImgNo::doc}}
\begin{sphinxuseclass}{cell}\begin{sphinxVerbatimInput}

\begin{sphinxuseclass}{cell_input}
\begin{sphinxVerbatim}[commandchars=\\\{\}]
\PYG{k+kn}{import} \PYG{n+nn}{lib}\PYG{n+nn}{.}\PYG{n+nn}{main}\PYG{n+nn}{.}\PYG{n+nn}{e5\PYGZus{}i\PYGZus{}anim\PYGZus{}hyperplot}

\PYG{n+nb}{print}\PYG{p}{(}\PYG{n}{lib}\PYG{o}{.}\PYG{n}{main}\PYG{o}{.}\PYG{n}{e5\PYGZus{}i\PYGZus{}anim\PYGZus{}hyperplot}\PYG{o}{.}\PYG{n}{lorentz\PYGZus{}t\PYGZus{}t00}\PYG{p}{(}\PYG{p}{)}\PYG{p}{)}
\PYG{n+nb}{print}\PYG{p}{(}\PYG{n}{lib}\PYG{o}{.}\PYG{n}{main}\PYG{o}{.}\PYG{n}{e5\PYGZus{}i\PYGZus{}anim\PYGZus{}hyperplot}\PYG{o}{.}\PYG{n}{lorentz\PYGZus{}t\PYGZus{}f00}\PYG{p}{(}\PYG{p}{)}\PYG{p}{)}
\PYG{n}{lib}\PYG{o}{.}\PYG{n}{main}\PYG{o}{.}\PYG{n}{e5\PYGZus{}i\PYGZus{}anim\PYGZus{}hyperplot}\PYG{o}{.}\PYG{n}{hyperplot\PYGZus{}and\PYGZus{}tell}\PYG{p}{(}\PYG{p}{)}
\end{sphinxVerbatim}

\end{sphinxuseclass}\end{sphinxVerbatimInput}
\begin{sphinxVerbatimOutput}

\begin{sphinxuseclass}{cell_output}
\begin{sphinxVerbatim}[commandchars=\\\{\}]
lib.main.e5\PYGZus{}i\PYGZus{}anim\PYGZus{}hyperplot
not via e5 main
[0, 6.928203230275509]
[\PYGZhy{}4.618802153517006, 9.237604307034012]
\end{sphinxVerbatim}

\noindent\sphinxincludegraphics{{32dd89aee47f7a4cb3bd512d5d86b2164c4d35d181a301574cde9ebcf68b76ea}.png}

\end{sphinxuseclass}\end{sphinxVerbatimOutput}

\end{sphinxuseclass}
\begin{sphinxVerbatim}[commandchars=\\\{\}]
A taste of part 2

\PYGZhy{} if quantum mechanic is about the rotation, one observe is that i does not change during rotation
\PYGZhy{} the reason is that its magnitude is defined for a + bi \PYGZhy{}\PYGZgt{} (a + bi)(a\PYGZhy{}bi) or a\PYGZca{}2 + b\PYGZca{}2

What if we invent another imaginery number call split complex number where

j\PYGZca{}2 = +1! 

Well you say it is pointless as j is just 1.  But once we accept you can have 2 dimension ... why not?

The issue is that what this imaginery number rotate.  The magnitude should be constant and hence

A = (a + bj)(a\PYGZhy{}bj) = a\PYGZca{}2 \PYGZhy{} b\PYGZca{}2. Note the \PYGZhy{}ve sign there.

It is a hyperbola not a circle it will be draw when it rotate.

Use?

Well, it turns out our spacetime are related like this

(ct)\PYGZca{}2 \PYGZhy{} x\PYGZca{}2 \PYGZhy{} y\PYGZca{}2 \PYGZhy{} z\PYGZca{}2 or if concentrate on only time and say x dimension movement it would be

a\PYGZca{}2 \PYGZhy{} b\PYGZca{}2 format

all those talk about time dilution and length contraction ... all because of this split complex number
\end{sphinxVerbatim}

\begin{sphinxVerbatim}[commandchars=\\\{\}]
Not only we live in imaginery number as a quantum being 
We also live in split imaginery number as a relativity being

I am not sure I want to got the third imaginery number well ... 
\end{sphinxVerbatim}

\sphinxAtStartPar
Part 2 (to be developed)

\sphinxAtStartPar
—> Draft to be tidy up with demo

\sphinxAtStartPar
– what if our world use this split complex number as above using j\textasciicircum{}2 = 1 (instead of i\textasciicircum{}2 = \sphinxhyphen{}1)

\sphinxAtStartPar
Unlike Galieo and Netwon which use absolute time and space and hence change observation frame by traveling say on a train does not affect the time and space.  You worldline shift (and not rotate) … but this is against the issue that one thing does not shift or light speed cannot be changed.

\sphinxAtStartPar
Here the invariant is not time or space but a join number by them (or split complex number).

\sphinxAtStartPar
More importantly the light wave has zero spllit complex number and hence whatever you “rotate” its speed does not affect because it is 0.  0*whatsoever is still 0.

\sphinxAtStartPar
Or 0 = x+tj (here j is in time and s in distance, with x\textasciicircum{}2 \sphinxhyphen{} t\textasciicircum{}2 = constant)

\sphinxAtStartPar
In fact solve this equation x = +/\sphinxhyphen{}t and by using a proper light always travel at  45 degree (or pi/4 radians) and 135 degree … i.e. just x = t and x = \sphinxhyphen{}t  (scale \sphinxhyphen{} t in year say and x is distance travel by light in 1 year)

\sphinxAtStartPar
Graphic (see above)


\bigskip\hrule\bigskip


\sphinxAtStartPar
To understand

\sphinxAtStartPar
Yes it helps a bit in understand one of the strange thing about relativity.  The key issue is about the

\sphinxAtStartPar
Galielo relativlity (the shift not rotate of timeline)

\sphinxAtStartPar
Graphic

\sphinxAtStartPar
and

\sphinxAtStartPar
The strange way our real world operate or there is a constant that is independent from our frame of reference (at  0 or constant speed).  0 set the limit and non\sphinxhyphen{}0 confine us to < c.

\sphinxAtStartPar
–

\sphinxAtStartPar
Graphic see

\sphinxAtStartPar
One observer (red, point upward as x=0 and t ever\sphinxhyphen{}increasing)
Two observer (red + green)
View from first (red point upward and green to the right)
View from second (red point to the left and green now point upwards as at rest)
Note the rotation is using hyperbola as whatever frame rotate there is a constant s\textasciicircum{}2 = t\textasciicircum{}2 \sphinxhyphen{} x\textasciicircum{}2

\sphinxAtStartPar
And most important if you just look at time, you can see it shorten and so is length.  Becuase it is the difference between time and space that matters.  Individual dimension it change this way becaues of the negative side.  (BTW, some textbook do  x\textasciicircum{}2 \sphinxhyphen{} t\textasciicircum{}2.  )

\sphinxAtStartPar
Part 2 end
\begin{enumerate}
\sphinxsetlistlabels{\arabic}{enumi}{enumii}{}{.}%
\item {} 
\sphinxAtStartPar
Twin Paradox (a third blue observer, acceleration or doppler)

\item {} 
\sphinxAtStartPar
Geometry change from inerteria to acceleration or Rindler not Gravity

\item {} 
\sphinxAtStartPar
Then how about the whole geometry from flat to curved due to mass
(General, using projective and differntial geomtry, plus group …)
(Not about acceleration as in 2, as gravity is not acceleration)

\item {} 
\sphinxAtStartPar
other imaginery or complex  number like dual number

\end{enumerate}

\sphinxAtStartPar
<— Draft to be tidy up with demo

\begin{sphinxVerbatim}[commandchars=\\\{\}]
Finaly, the complex number and split complex number has joined as Quantum Field Theory
Or the wave travel in relativity manner, 
\PYGZhy{}\PYGZhy{} its vibration follow the complex number 
\PYGZhy{}\PYGZhy{} but its velociy follow the split complex number (a hyperbola worldline).  
BTW, sadly that is already 50+ years ago.

Unfortunately we still do not have the wave onto the general relativity
    \PYGZhy{}\PYGZhy{} projective geomoetry and differential geometry ...

Thanks for your patient and attention.

We live briefly in these 2 imaginery number (and more) world.  

What you see is NOT real or at least NOT JUST REAL!
\end{sphinxVerbatim}

\sphinxstepscope


\chapter{Appendix \sphinxhyphen{} may be needed for reference}
\label{\detokenize{MacTwgssAx-App:appendix-may-be-needed-for-reference}}\label{\detokenize{MacTwgssAx-App::doc}}
\begin{sphinxuseclass}{cell}\begin{sphinxVerbatimInput}

\begin{sphinxuseclass}{cell_input}
\begin{sphinxVerbatim}[commandchars=\\\{\}]
\PYG{k+kn}{import} \PYG{n+nn}{os}

\PYG{n+nb}{print}\PYG{p}{(}\PYG{n}{os}\PYG{o}{.}\PYG{n}{getcwd}\PYG{p}{(}\PYG{p}{)}\PYG{p}{)}
\end{sphinxVerbatim}

\end{sphinxuseclass}\end{sphinxVerbatimInput}
\begin{sphinxVerbatimOutput}

\begin{sphinxuseclass}{cell_output}
\begin{sphinxVerbatim}[commandchars=\\\{\}]
/Users/ngcchk/Documents/GitHub/gpd2\PYGZhy{}win\PYGZhy{}unity1/ipadred\PYGZhy{}rain/imgno\PYGZus{}book1/imgnobk1
\end{sphinxVerbatim}

\end{sphinxuseclass}\end{sphinxVerbatimOutput}

\end{sphinxuseclass}
\begin{sphinxuseclass}{cell}\begin{sphinxVerbatimInput}

\begin{sphinxuseclass}{cell_input}
\begin{sphinxVerbatim}[commandchars=\\\{\}]
\PYG{k+kn}{import} \PYG{n+nn}{lib}\PYG{n+nn}{.}\PYG{n+nn}{main}\PYG{n+nn}{.}\PYG{n+nn}{a0\PYGZus{}babylon\PYGZus{}pos\PYGZus{}0}
\end{sphinxVerbatim}

\end{sphinxuseclass}\end{sphinxVerbatimInput}
\begin{sphinxVerbatimOutput}

\begin{sphinxuseclass}{cell_output}
\begin{sphinxVerbatim}[commandchars=\\\{\}]
not in main of a0
\end{sphinxVerbatim}

\end{sphinxuseclass}\end{sphinxVerbatimOutput}

\end{sphinxuseclass}
\begin{sphinxuseclass}{cell}\begin{sphinxVerbatimInput}

\begin{sphinxuseclass}{cell_input}
\begin{sphinxVerbatim}[commandchars=\\\{\}]
\PYG{n}{lib}\PYG{o}{.}\PYG{n}{main}\PYG{o}{.}\PYG{n}{a0\PYGZus{}babylon\PYGZus{}pos\PYGZus{}0}\PYG{o}{.}\PYG{n}{display\PYGZus{}img}\PYG{p}{(}\PYG{l+m+mi}{1}\PYG{p}{)}
\end{sphinxVerbatim}

\end{sphinxuseclass}\end{sphinxVerbatimInput}
\begin{sphinxVerbatimOutput}

\begin{sphinxuseclass}{cell_output}
\noindent\sphinxincludegraphics{{471baaa0194945d1ab434b498e8531a1c6ea386bbd7c0c99339311f2abe37783}.png}

\end{sphinxuseclass}\end{sphinxVerbatimOutput}

\end{sphinxuseclass}
\begin{sphinxVerbatim}[commandchars=\\\{\}]
        note 
        a) there are only 2 symbols 1 and 10
        b) every number is actually by placing and counting number 1 and number 10 out; FULLY
        c) reaching 59 then what ... 60 is a problem let us skip it first ;\PYGZhy{}P
\end{sphinxVerbatim}

\begin{sphinxuseclass}{cell}\begin{sphinxVerbatimInput}

\begin{sphinxuseclass}{cell_input}
\begin{sphinxVerbatim}[commandchars=\\\{\}]
\PYG{n}{lib}\PYG{o}{.}\PYG{n}{main}\PYG{o}{.}\PYG{n}{a0\PYGZus{}babylon\PYGZus{}pos\PYGZus{}0}\PYG{o}{.}\PYG{n}{display\PYGZus{}img}\PYG{p}{(}\PYG{l+m+mi}{2}\PYG{p}{)}
\end{sphinxVerbatim}

\end{sphinxuseclass}\end{sphinxVerbatimInput}
\begin{sphinxVerbatimOutput}

\begin{sphinxuseclass}{cell_output}
\noindent\sphinxincludegraphics{{39571f3cb2beee813dda02e06fe31aaa3beda6e9523d43a835a0beb6c571cc13}.png}

\end{sphinxuseclass}\end{sphinxVerbatimOutput}

\end{sphinxuseclass}
\begin{sphinxuseclass}{cell}\begin{sphinxVerbatimInput}

\begin{sphinxuseclass}{cell_input}
\begin{sphinxVerbatim}[commandchars=\\\{\}]
\PYG{n}{lib}\PYG{o}{.}\PYG{n}{main}\PYG{o}{.}\PYG{n}{a0\PYGZus{}babylon\PYGZus{}pos\PYGZus{}0}\PYG{o}{.}\PYG{n}{display\PYGZus{}img}\PYG{p}{(}\PYG{l+m+mi}{3}\PYG{p}{)}
\end{sphinxVerbatim}

\end{sphinxuseclass}\end{sphinxVerbatimInput}
\begin{sphinxVerbatimOutput}

\begin{sphinxuseclass}{cell_output}
\noindent\sphinxincludegraphics{{ee9ec6e7b82ed8a727bce0a03d5c809f3e8773a8ced5fc0012603ae2cbd4a6e7}.png}

\end{sphinxuseclass}\end{sphinxVerbatimOutput}

\end{sphinxuseclass}
\begin{sphinxuseclass}{cell}\begin{sphinxVerbatimInput}

\begin{sphinxuseclass}{cell_input}
\begin{sphinxVerbatim}[commandchars=\\\{\}]
\PYG{o}{\PYGZpc{}}\PYG{k}{matplotlib} inline
\PYG{k+kn}{import} \PYG{n+nn}{lib}\PYG{n+nn}{.}\PYG{n+nn}{main}\PYG{n+nn}{.}\PYG{n+nn}{a1\PYGZus{}i\PYGZus{}mul\PYGZus{}plot}
\end{sphinxVerbatim}

\end{sphinxuseclass}\end{sphinxVerbatimInput}
\begin{sphinxVerbatimOutput}

\begin{sphinxuseclass}{cell_output}
\begin{sphinxVerbatim}[commandchars=\\\{\}]
lib.main.a1\PYGZus{}i\PYGZus{}mul\PYGZus{}plot
not via main
it is a rotation
and i always rotate 90 degree or pi/2 radians
\end{sphinxVerbatim}

\noindent\sphinxincludegraphics{{f63e9dabeacd2459c5b43c505f880985e3bac2a9972f27577770d543149abeff}.png}

\end{sphinxuseclass}\end{sphinxVerbatimOutput}

\end{sphinxuseclass}
\begin{sphinxuseclass}{cell}\begin{sphinxVerbatimInput}

\begin{sphinxuseclass}{cell_input}
\begin{sphinxVerbatim}[commandchars=\\\{\}]
\PYG{o}{\PYGZpc{}}\PYG{k}{matplotlib} inline
\PYG{n}{lib}\PYG{o}{.}\PYG{n}{main}\PYG{o}{.}\PYG{n}{a1\PYGZus{}i\PYGZus{}mul\PYGZus{}plot}\PYG{o}{.}\PYG{n}{some\PYGZus{}plot}\PYG{p}{(}\PYG{p}{)}
\end{sphinxVerbatim}

\end{sphinxuseclass}\end{sphinxVerbatimInput}
\begin{sphinxVerbatimOutput}

\begin{sphinxuseclass}{cell_output}
\begin{sphinxVerbatim}[commandchars=\\\{\}]
\PYGZhy{}\PYGZhy{} more example\PYGZhy{}\PYGZhy{}
\end{sphinxVerbatim}

\noindent\sphinxincludegraphics{{9ca01360ece7cb0f6ae239c6c5012e88cf0789811fb1cd49ba3b5777042089a7}.png}

\noindent\sphinxincludegraphics{{1d5ca392f2e0f54d998cd34b2d2870c98a5d4ac536eb8c8d970717b62ff8c8ba}.png}

\noindent\sphinxincludegraphics{{0fec07713bc70a91cfdfe4060246b271e12c72622281031e5fe466f8ae10ca6a}.png}

\noindent\sphinxincludegraphics{{2eb9c53133a9d2d01655f13ca2c5d6f787a8bed41fc77dc560ce5d5df7baf655}.png}

\noindent\sphinxincludegraphics{{253e81a4e934cbfa4f42f4c626fd661a9986be6bb6d6da961912efdb7d3cc50a}.png}

\noindent\sphinxincludegraphics{{f63e9dabeacd2459c5b43c505f880985e3bac2a9972f27577770d543149abeff}.png}

\noindent\sphinxincludegraphics{{79f79dc8d0bb4c3553822d24147d965a264fddf2af996a19d335228e07073390}.png}

\noindent\sphinxincludegraphics{{8ff538469fc018881da1b25af26f1b55ce9889795eb6c7f6b84fcd2dd99e589c}.png}

\noindent\sphinxincludegraphics{{9ca01360ece7cb0f6ae239c6c5012e88cf0789811fb1cd49ba3b5777042089a7}.png}

\end{sphinxuseclass}\end{sphinxVerbatimOutput}

\end{sphinxuseclass}
\begin{sphinxuseclass}{cell}\begin{sphinxVerbatimInput}

\begin{sphinxuseclass}{cell_input}
\begin{sphinxVerbatim}[commandchars=\\\{\}]
\PYG{c+c1}{\PYGZsh{}import lib.main.e5\PYGZus{}i\PYGZus{}anim\PYGZus{}hyperplot }
\PYG{n+nb}{print} \PYG{p}{(}\PYG{l+s+s2}{\PYGZdq{}}\PYG{l+s+s2}{c3; red start and blue end}\PYG{l+s+s2}{\PYGZdq{}}\PYG{p}{)}
\PYG{k+kn}{import} \PYG{n+nn}{lib}\PYG{n+nn}{.}\PYG{n+nn}{main}\PYG{n+nn}{.}\PYG{n+nn}{c3\PYGZus{}i\PYGZus{}anim\PYGZus{}circle\PYGZus{}rotate\PYGZus{}plot}
\PYG{k+kn}{import} \PYG{n+nn}{numpy} \PYG{k}{as} \PYG{n+nn}{np}
\PYG{n}{lib}\PYG{o}{.}\PYG{n}{main}\PYG{o}{.}\PYG{n}{c3\PYGZus{}i\PYGZus{}anim\PYGZus{}circle\PYGZus{}rotate\PYGZus{}plot}\PYG{o}{.}\PYG{n}{plot\PYGZus{}i\PYGZus{}circle}\PYG{p}{(}\PYG{n}{np}\PYG{o}{.}\PYG{n}{pi}\PYG{o}{/}\PYG{l+m+mi}{4}\PYG{p}{)}
\end{sphinxVerbatim}

\end{sphinxuseclass}\end{sphinxVerbatimInput}
\begin{sphinxVerbatimOutput}

\begin{sphinxuseclass}{cell_output}
\begin{sphinxVerbatim}[commandchars=\\\{\}]
c3; red start and blue end
lib.main.e5\PYGZus{}i\PYGZus{}anim\PYGZus{}hyperplot
not via e5 main
/Users/ngcchk/Documents/GitHub/gpd2\PYGZhy{}win\PYGZhy{}unity1/ipadred\PYGZhy{}rain/imgno\PYGZus{}book1/imgnobk1
not vai c3 main

index: 27 
theta4 at  4.554255121982519 
 a= \PYGZhy{}0.15747562437201815  b= \PYGZhy{}0.9875228745343791 
\end{sphinxVerbatim}

\noindent\sphinxincludegraphics{{b151346207bfcf4247e177aa1da743dbc40b210fa68ed2b55c930d82a7006352}.png}

\end{sphinxuseclass}\end{sphinxVerbatimOutput}

\end{sphinxuseclass}
\begin{sphinxuseclass}{cell}\begin{sphinxVerbatimInput}

\begin{sphinxuseclass}{cell_input}
\begin{sphinxVerbatim}[commandchars=\\\{\}]
\PYG{n+nb}{print} \PYG{p}{(}\PYG{l+s+s2}{\PYGZdq{}}\PYG{l+s+s2}{another example \PYGZhy{} c3; red start and blue end}\PYG{l+s+s2}{\PYGZdq{}}\PYG{p}{)}
\PYG{k+kn}{import} \PYG{n+nn}{lib}\PYG{n+nn}{.}\PYG{n+nn}{main}\PYG{n+nn}{.}\PYG{n+nn}{c3\PYGZus{}i\PYGZus{}anim\PYGZus{}circle\PYGZus{}rotate\PYGZus{}plot}
\PYG{k+kn}{import} \PYG{n+nn}{numpy} \PYG{k}{as} \PYG{n+nn}{np}
\PYG{c+c1}{\PYGZsh{}lib.main.c3\PYGZus{}i\PYGZus{}anim\PYGZus{}circle\PYGZus{}rotate\PYGZus{}plot.do\PYGZus{}anim()}
\PYG{c+c1}{\PYGZsh{} need to run in terminal for the moment???}
\end{sphinxVerbatim}

\end{sphinxuseclass}\end{sphinxVerbatimInput}
\begin{sphinxVerbatimOutput}

\begin{sphinxuseclass}{cell_output}
\begin{sphinxVerbatim}[commandchars=\\\{\}]
another example \PYGZhy{} c3; red start and blue end
\end{sphinxVerbatim}

\end{sphinxuseclass}\end{sphinxVerbatimOutput}

\end{sphinxuseclass}
\begin{sphinxuseclass}{cell}\begin{sphinxVerbatimInput}

\begin{sphinxuseclass}{cell_input}
\begin{sphinxVerbatim}[commandchars=\\\{\}]
\PYG{k+kn}{import} \PYG{n+nn}{lib}\PYG{n+nn}{.}\PYG{n+nn}{main}\PYG{n+nn}{.}\PYG{n+nn}{e5\PYGZus{}i\PYGZus{}anim\PYGZus{}hyperplot}

\PYG{n+nb}{print}\PYG{p}{(}\PYG{n}{lib}\PYG{o}{.}\PYG{n}{main}\PYG{o}{.}\PYG{n}{e5\PYGZus{}i\PYGZus{}anim\PYGZus{}hyperplot}\PYG{o}{.}\PYG{n}{lorentz\PYGZus{}t\PYGZus{}t00}\PYG{p}{(}\PYG{p}{)}\PYG{p}{)}
\PYG{n+nb}{print}\PYG{p}{(}\PYG{n}{lib}\PYG{o}{.}\PYG{n}{main}\PYG{o}{.}\PYG{n}{e5\PYGZus{}i\PYGZus{}anim\PYGZus{}hyperplot}\PYG{o}{.}\PYG{n}{lorentz\PYGZus{}t\PYGZus{}f00}\PYG{p}{(}\PYG{p}{)}\PYG{p}{)}
\PYG{n}{lib}\PYG{o}{.}\PYG{n}{main}\PYG{o}{.}\PYG{n}{e5\PYGZus{}i\PYGZus{}anim\PYGZus{}hyperplot}\PYG{o}{.}\PYG{n}{hyperplot\PYGZus{}and\PYGZus{}tell}\PYG{p}{(}\PYG{p}{)}
\end{sphinxVerbatim}

\end{sphinxuseclass}\end{sphinxVerbatimInput}
\begin{sphinxVerbatimOutput}

\begin{sphinxuseclass}{cell_output}
\begin{sphinxVerbatim}[commandchars=\\\{\}]
[0, 6.928203230275509]
[\PYGZhy{}4.618802153517006, 9.237604307034012]
\end{sphinxVerbatim}

\noindent\sphinxincludegraphics{{32dd89aee47f7a4cb3bd512d5d86b2164c4d35d181a301574cde9ebcf68b76ea}.png}

\end{sphinxuseclass}\end{sphinxVerbatimOutput}

\end{sphinxuseclass}
\sphinxstepscope


\chapter{Welcome to your Jupyter Book}
\label{\detokenize{intro:welcome-to-your-jupyter-book}}\label{\detokenize{intro::doc}}
\sphinxAtStartPar
This is a small sample book to give you a feel for how book content is
structured.
It shows off a few of the major file types, as well as some sample content.
It does not go in\sphinxhyphen{}depth into any particular topic \sphinxhyphen{} check out \sphinxhref{https://jupyterbook.org}{the Jupyter Book documentation} for more information. See also \sphinxhref{https://jupyterbook.org/en/stable/start/create.html}{Jupter installtion}

\sphinxAtStartPar
Check out the content pages bundled with this sample book to see more.


\chapter{Page title}
\label{\detokenize{intro:page-title}}
\begin{sphinxShadowBox}
\begin{itemize}
\item {} 
\sphinxAtStartPar
\phantomsection\label{\detokenize{intro:id1}}{\hyperref[\detokenize{intro:section-1-will-be-listed}]{\sphinxcrossref{Section 1 (will be listed)}}}
\begin{itemize}
\item {} 
\sphinxAtStartPar
\phantomsection\label{\detokenize{intro:id2}}{\hyperref[\detokenize{intro:sub-section-1-will-be-listed}]{\sphinxcrossref{Sub\sphinxhyphen{}section 1 (will be listed)}}}

\end{itemize}

\item {} 
\sphinxAtStartPar
\phantomsection\label{\detokenize{intro:id3}}{\hyperref[\detokenize{intro:section-2-will-be-listed}]{\sphinxcrossref{Section 2 (will be listed)}}}

\end{itemize}
\end{sphinxShadowBox}


\section{Section 1 (will be listed)}
\label{\detokenize{intro:section-1-will-be-listed}}

\subsection{Sub\sphinxhyphen{}section 1 (will be listed)}
\label{\detokenize{intro:sub-section-1-will-be-listed}}

\section{Section 2 (will be listed)}
\label{\detokenize{intro:section-2-will-be-listed}}
\sphinxstepscope


\chapter{Markdown Files}
\label{\detokenize{markdown:markdown-files}}\label{\detokenize{markdown::doc}}
\sphinxAtStartPar
Whether you write your book’s content in Jupyter Notebooks (\sphinxcode{\sphinxupquote{.ipynb}}) or
in regular markdown files (\sphinxcode{\sphinxupquote{.md}}), you’ll write in the same flavor of markdown
called \sphinxstylestrong{MyST Markdown}.
This is a simple file to help you get started and show off some syntax.


\section{What is MyST?}
\label{\detokenize{markdown:what-is-myst}}
\sphinxAtStartPar
MyST stands for “Markedly Structured Text”. It
is a slight variation on a flavor of markdown called “CommonMark” markdown,
with small syntax extensions to allow you to write \sphinxstylestrong{roles} and \sphinxstylestrong{directives}
in the Sphinx ecosystem.

\sphinxAtStartPar
For more about MyST, see \sphinxhref{https://jupyterbook.org/content/myst.html}{the MyST Markdown Overview}.


\section{Sample Roles and Directives}
\label{\detokenize{markdown:sample-roles-and-directives}}
\sphinxAtStartPar
Roles and directives are two of the most powerful tools in Jupyter Book. They
are kind of like functions, but written in a markup language. They both
serve a similar purpose, but \sphinxstylestrong{roles are written in one line}, whereas
\sphinxstylestrong{directives span many lines}. They both accept different kinds of inputs,
and what they do with those inputs depends on the specific role or directive
that is being called.

\sphinxAtStartPar
Here is a “note” directive:

\begin{sphinxadmonition}{note}{Note:}
\sphinxAtStartPar
Here is a note
\end{sphinxadmonition}

\sphinxAtStartPar
It will be rendered in a special box when you build your book.

\sphinxAtStartPar
Here is an inline directive to refer to a document: {\hyperref[\detokenize{markdown-notebooks::doc}]{\sphinxcrossref{\DUrole{doc}{Notebooks with MyST Markdown}}}}.


\section{Citations}
\label{\detokenize{markdown:citations}}
\sphinxAtStartPar
You can also cite references that are stored in a \sphinxcode{\sphinxupquote{bibtex}} file. For example,
the following syntax: \sphinxcode{\sphinxupquote{\{cite\}`holdgraf\_evidence\_2014`}} will render like
this: {[}\hyperlink{cite.markdown:id3}{HdHPK14}{]}.

\sphinxAtStartPar
Moreover, you can insert a bibliography into your page with this syntax:
The \sphinxcode{\sphinxupquote{\{bibliography\}}} directive must be used for all the \sphinxcode{\sphinxupquote{\{cite\}}} roles to
render properly.
For example, if the references for your book are stored in \sphinxcode{\sphinxupquote{references.bib}},
then the bibliography is inserted with:


\section{Learn more}
\label{\detokenize{markdown:learn-more}}
\sphinxAtStartPar
This is just a simple starter to get you started.
You can learn a lot more at \sphinxhref{https://jupyterbook.org}{jupyterbook.org}.

\sphinxstepscope


\chapter{Content with notebooks}
\label{\detokenize{notebooks:content-with-notebooks}}\label{\detokenize{notebooks::doc}}
\sphinxAtStartPar
You can also create content with Jupyter Notebooks. This means that you can include
code blocks and their outputs in your book.


\section{Markdown + notebooks}
\label{\detokenize{notebooks:markdown-notebooks}}
\sphinxAtStartPar
As it is markdown, you can embed images, HTML, etc into your posts!

\sphinxAtStartPar
\sphinxincludegraphics{{logo-wide}.png}

\sphinxAtStartPar
You can also \(add_{math}\) and
\begin{equation*}
\begin{split}
math^{blocks}
\end{split}
\end{equation*}
\sphinxAtStartPar
or
\begin{equation*}
\begin{split}
\begin{aligned}
\mbox{mean} la_{tex} \\ \\
math blocks
\end{aligned}
\end{split}
\end{equation*}
\sphinxAtStartPar
But make sure you \$Escape \$your \$dollar signs \$you want to keep!


\section{MyST markdown}
\label{\detokenize{notebooks:myst-markdown}}
\sphinxAtStartPar
MyST markdown works in Jupyter Notebooks as well. For more information about MyST markdown, check
out \sphinxhref{https://jupyterbook.org/content/myst.html}{the MyST guide in Jupyter Book},
or see \sphinxhref{https://myst-parser.readthedocs.io/en/latest/}{the MyST markdown documentation}.


\section{Code blocks and outputs}
\label{\detokenize{notebooks:code-blocks-and-outputs}}
\sphinxAtStartPar
Jupyter Book will also embed your code blocks and output in your book.
For example, here’s some sample Matplotlib code:

\begin{sphinxuseclass}{cell}\begin{sphinxVerbatimInput}

\begin{sphinxuseclass}{cell_input}
\begin{sphinxVerbatim}[commandchars=\\\{\}]
\PYG{k+kn}{from} \PYG{n+nn}{matplotlib} \PYG{k+kn}{import} \PYG{n}{rcParams}\PYG{p}{,} \PYG{n}{cycler}
\PYG{k+kn}{import} \PYG{n+nn}{matplotlib}\PYG{n+nn}{.}\PYG{n+nn}{pyplot} \PYG{k}{as} \PYG{n+nn}{plt}
\PYG{k+kn}{import} \PYG{n+nn}{numpy} \PYG{k}{as} \PYG{n+nn}{np}
\PYG{n}{plt}\PYG{o}{.}\PYG{n}{ion}\PYG{p}{(}\PYG{p}{)}
\end{sphinxVerbatim}

\end{sphinxuseclass}\end{sphinxVerbatimInput}
\begin{sphinxVerbatimOutput}

\begin{sphinxuseclass}{cell_output}
\begin{sphinxVerbatim}[commandchars=\\\{\}]
\PYGZlt{}contextlib.ExitStack at 0x126c05610\PYGZgt{}
\end{sphinxVerbatim}

\end{sphinxuseclass}\end{sphinxVerbatimOutput}

\end{sphinxuseclass}
\begin{sphinxuseclass}{cell}\begin{sphinxVerbatimInput}

\begin{sphinxuseclass}{cell_input}
\begin{sphinxVerbatim}[commandchars=\\\{\}]
\PYG{c+c1}{\PYGZsh{} Fixing random state for reproducibility}
\PYG{n}{np}\PYG{o}{.}\PYG{n}{random}\PYG{o}{.}\PYG{n}{seed}\PYG{p}{(}\PYG{l+m+mi}{19680801}\PYG{p}{)}

\PYG{n}{N} \PYG{o}{=} \PYG{l+m+mi}{10}
\PYG{n}{data} \PYG{o}{=} \PYG{p}{[}\PYG{n}{np}\PYG{o}{.}\PYG{n}{logspace}\PYG{p}{(}\PYG{l+m+mi}{0}\PYG{p}{,} \PYG{l+m+mi}{1}\PYG{p}{,} \PYG{l+m+mi}{100}\PYG{p}{)} \PYG{o}{+} \PYG{n}{np}\PYG{o}{.}\PYG{n}{random}\PYG{o}{.}\PYG{n}{randn}\PYG{p}{(}\PYG{l+m+mi}{100}\PYG{p}{)} \PYG{o}{+} \PYG{n}{ii} \PYG{k}{for} \PYG{n}{ii} \PYG{o+ow}{in} \PYG{n+nb}{range}\PYG{p}{(}\PYG{n}{N}\PYG{p}{)}\PYG{p}{]}
\PYG{n}{data} \PYG{o}{=} \PYG{n}{np}\PYG{o}{.}\PYG{n}{array}\PYG{p}{(}\PYG{n}{data}\PYG{p}{)}\PYG{o}{.}\PYG{n}{T}
\PYG{n}{cmap} \PYG{o}{=} \PYG{n}{plt}\PYG{o}{.}\PYG{n}{cm}\PYG{o}{.}\PYG{n}{coolwarm}
\PYG{n}{rcParams}\PYG{p}{[}\PYG{l+s+s1}{\PYGZsq{}}\PYG{l+s+s1}{axes.prop\PYGZus{}cycle}\PYG{l+s+s1}{\PYGZsq{}}\PYG{p}{]} \PYG{o}{=} \PYG{n}{cycler}\PYG{p}{(}\PYG{n}{color}\PYG{o}{=}\PYG{n}{cmap}\PYG{p}{(}\PYG{n}{np}\PYG{o}{.}\PYG{n}{linspace}\PYG{p}{(}\PYG{l+m+mi}{0}\PYG{p}{,} \PYG{l+m+mi}{1}\PYG{p}{,} \PYG{n}{N}\PYG{p}{)}\PYG{p}{)}\PYG{p}{)}


\PYG{k+kn}{from} \PYG{n+nn}{matplotlib}\PYG{n+nn}{.}\PYG{n+nn}{lines} \PYG{k+kn}{import} \PYG{n}{Line2D}
\PYG{n}{custom\PYGZus{}lines} \PYG{o}{=} \PYG{p}{[}\PYG{n}{Line2D}\PYG{p}{(}\PYG{p}{[}\PYG{l+m+mi}{0}\PYG{p}{]}\PYG{p}{,} \PYG{p}{[}\PYG{l+m+mi}{0}\PYG{p}{]}\PYG{p}{,} \PYG{n}{color}\PYG{o}{=}\PYG{n}{cmap}\PYG{p}{(}\PYG{l+m+mf}{0.}\PYG{p}{)}\PYG{p}{,} \PYG{n}{lw}\PYG{o}{=}\PYG{l+m+mi}{4}\PYG{p}{)}\PYG{p}{,}
                \PYG{n}{Line2D}\PYG{p}{(}\PYG{p}{[}\PYG{l+m+mi}{0}\PYG{p}{]}\PYG{p}{,} \PYG{p}{[}\PYG{l+m+mi}{0}\PYG{p}{]}\PYG{p}{,} \PYG{n}{color}\PYG{o}{=}\PYG{n}{cmap}\PYG{p}{(}\PYG{l+m+mf}{.5}\PYG{p}{)}\PYG{p}{,} \PYG{n}{lw}\PYG{o}{=}\PYG{l+m+mi}{4}\PYG{p}{)}\PYG{p}{,}
                \PYG{n}{Line2D}\PYG{p}{(}\PYG{p}{[}\PYG{l+m+mi}{0}\PYG{p}{]}\PYG{p}{,} \PYG{p}{[}\PYG{l+m+mi}{0}\PYG{p}{]}\PYG{p}{,} \PYG{n}{color}\PYG{o}{=}\PYG{n}{cmap}\PYG{p}{(}\PYG{l+m+mf}{1.}\PYG{p}{)}\PYG{p}{,} \PYG{n}{lw}\PYG{o}{=}\PYG{l+m+mi}{4}\PYG{p}{)}\PYG{p}{]}

\PYG{n}{fig}\PYG{p}{,} \PYG{n}{ax} \PYG{o}{=} \PYG{n}{plt}\PYG{o}{.}\PYG{n}{subplots}\PYG{p}{(}\PYG{n}{figsize}\PYG{o}{=}\PYG{p}{(}\PYG{l+m+mi}{10}\PYG{p}{,} \PYG{l+m+mi}{5}\PYG{p}{)}\PYG{p}{)}
\PYG{n}{lines} \PYG{o}{=} \PYG{n}{ax}\PYG{o}{.}\PYG{n}{plot}\PYG{p}{(}\PYG{n}{data}\PYG{p}{)}
\PYG{n}{ax}\PYG{o}{.}\PYG{n}{legend}\PYG{p}{(}\PYG{n}{custom\PYGZus{}lines}\PYG{p}{,} \PYG{p}{[}\PYG{l+s+s1}{\PYGZsq{}}\PYG{l+s+s1}{Cold}\PYG{l+s+s1}{\PYGZsq{}}\PYG{p}{,} \PYG{l+s+s1}{\PYGZsq{}}\PYG{l+s+s1}{Medium}\PYG{l+s+s1}{\PYGZsq{}}\PYG{p}{,} \PYG{l+s+s1}{\PYGZsq{}}\PYG{l+s+s1}{Hot}\PYG{l+s+s1}{\PYGZsq{}}\PYG{p}{]}\PYG{p}{)}\PYG{p}{;}
\end{sphinxVerbatim}

\end{sphinxuseclass}\end{sphinxVerbatimInput}
\begin{sphinxVerbatimOutput}

\begin{sphinxuseclass}{cell_output}
\noindent\sphinxincludegraphics{{a5ce989de8ce057087826bc0e007fe74f08b1378d0934d78b54f3749bae0cc6c}.png}

\end{sphinxuseclass}\end{sphinxVerbatimOutput}

\end{sphinxuseclass}
\sphinxAtStartPar
There is a lot more that you can do with outputs (such as including interactive outputs)
with your book. For more information about this, see \sphinxhref{https://jupyterbook.org}{the Jupyter Book documentation}

\sphinxstepscope


\chapter{Notebooks with MyST Markdown}
\label{\detokenize{markdown-notebooks:notebooks-with-myst-markdown}}\label{\detokenize{markdown-notebooks::doc}}
\sphinxAtStartPar
Jupyter Book also lets you write text\sphinxhyphen{}based notebooks using MyST Markdown.
See \sphinxhref{https://jupyterbook.org/file-types/myst-notebooks.html}{the Notebooks with MyST Markdown documentation} for more detailed instructions.
This page shows off a notebook written in MyST Markdown.


\section{An example cell}
\label{\detokenize{markdown-notebooks:an-example-cell}}
\sphinxAtStartPar
With MyST Markdown, you can define code cells with a directive like so:

\begin{sphinxuseclass}{cell}\begin{sphinxVerbatimInput}

\begin{sphinxuseclass}{cell_input}
\begin{sphinxVerbatim}[commandchars=\\\{\}]
\PYG{n+nb}{print}\PYG{p}{(}\PYG{l+m+mi}{2} \PYG{o}{+} \PYG{l+m+mi}{2}\PYG{p}{)}
\end{sphinxVerbatim}

\end{sphinxuseclass}\end{sphinxVerbatimInput}
\begin{sphinxVerbatimOutput}

\begin{sphinxuseclass}{cell_output}
\begin{sphinxVerbatim}[commandchars=\\\{\}]
4
\end{sphinxVerbatim}

\end{sphinxuseclass}\end{sphinxVerbatimOutput}

\end{sphinxuseclass}
\sphinxAtStartPar
When your book is built, the contents of any \sphinxcode{\sphinxupquote{\{code\sphinxhyphen{}cell\}}} blocks will be
executed with your default Jupyter kernel, and their outputs will be displayed
in\sphinxhyphen{}line with the rest of your content.


\sphinxstrong{See also:}
\nopagebreak


\sphinxAtStartPar
Jupyter Book uses \sphinxhref{https://jupytext.readthedocs.io/en/latest/}{Jupytext} to convert text\sphinxhyphen{}based files to notebooks, and can support \sphinxhref{https://jupyterbook.org/file-types/jupytext.html}{many other text\sphinxhyphen{}based notebook files}.




\section{Create a notebook with MyST Markdown}
\label{\detokenize{markdown-notebooks:create-a-notebook-with-myst-markdown}}
\sphinxAtStartPar
MyST Markdown notebooks are defined by two things:
\begin{enumerate}
\sphinxsetlistlabels{\arabic}{enumi}{enumii}{}{.}%
\item {} 
\sphinxAtStartPar
YAML metadata that is needed to understand if / how it should convert text files to notebooks (including information about the kernel needed).
See the YAML at the top of this page for example.

\item {} 
\sphinxAtStartPar
The presence of \sphinxcode{\sphinxupquote{\{code\sphinxhyphen{}cell\}}} directives, which will be executed with your book.

\end{enumerate}

\sphinxAtStartPar
That’s all that is needed to get started!


\section{Quickly add YAML metadata for MyST Notebooks}
\label{\detokenize{markdown-notebooks:quickly-add-yaml-metadata-for-myst-notebooks}}
\sphinxAtStartPar
If you have a markdown file and you’d like to quickly add YAML metadata to it, so that Jupyter Book will treat it as a MyST Markdown Notebook, run the following command:

\begin{sphinxVerbatim}[commandchars=\\\{\}]
\PYG{n}{jupyter}\PYG{o}{\PYGZhy{}}\PYG{n}{book} \PYG{n}{myst} \PYG{n}{init} \PYG{n}{path}\PYG{o}{/}\PYG{n}{to}\PYG{o}{/}\PYG{n}{markdownfile}\PYG{o}{.}\PYG{n}{md}
\end{sphinxVerbatim}

\sphinxstepscope


\chapter{testing jupyter\sphinxhyphen{}book failed using fAe examples}
\label{\detokenize{fAe:testing-jupyter-book-failed-using-fae-examples}}\label{\detokenize{fAe::doc}}
\sphinxAtStartPar
from \sphinxurl{https://github.com/markjay4k/fourier-transform}
unless matplotlib notebook to inline

\sphinxAtStartPar
Not work immeidately

\sphinxAtStartPar
as jupyterlab assume using conda and use javascript for
interaction … that is not in the past \sphinxurl{https://github.com/jupyterlab/jupyterlab/issues/3934}

\sphinxAtStartPar
as detailed discussed here: \sphinxurl{https://stackoverflow.com/questions/51922480/javascript-error-ipython-is-not-defined-in-jupyterlab/56416229\#56416229}

\sphinxAtStartPar
change all to inline it would work but what is the point???!!!!

\begin{sphinxuseclass}{cell}\begin{sphinxVerbatimInput}

\begin{sphinxuseclass}{cell_input}
\begin{sphinxVerbatim}[commandchars=\\\{\}]
\PYG{k+kn}{import} \PYG{n+nn}{numpy} \PYG{k}{as} \PYG{n+nn}{np}

\PYG{k}{def} \PYG{n+nf}{rect}\PYG{p}{(}\PYG{n}{x}\PYG{p}{,} \PYG{n}{B}\PYG{p}{)}\PYG{p}{:}
\PYG{+w}{    }\PYG{l+s+sd}{\PYGZdq{}\PYGZdq{}\PYGZdq{}}
\PYG{l+s+sd}{    create a rectangle function}
\PYG{l+s+sd}{    returns a numpy array that is 1 if |x| \PYGZlt{} w and 0 if |x| \PYGZgt{} w}
\PYG{l+s+sd}{    B is the rectangle width centered at 0}
\PYG{l+s+sd}{    x is the number of points in the array}
\PYG{l+s+sd}{    \PYGZdq{}\PYGZdq{}\PYGZdq{}}
    
    \PYG{n}{B} \PYG{o}{=} \PYG{n+nb}{int}\PYG{p}{(}\PYG{n}{B}\PYG{p}{)}
    \PYG{n}{x} \PYG{o}{=} \PYG{n+nb}{int}\PYG{p}{(}\PYG{n}{x}\PYG{p}{)}
    
    \PYG{n}{high} \PYG{o}{=} \PYG{n}{np}\PYG{o}{.}\PYG{n}{ones}\PYG{p}{(}\PYG{n}{B}\PYG{p}{)}
    \PYG{n}{low1} \PYG{o}{=} \PYG{n}{np}\PYG{o}{.}\PYG{n}{zeros}\PYG{p}{(}\PYG{n+nb}{int}\PYG{p}{(}\PYG{n}{x}\PYG{o}{/}\PYG{l+m+mi}{2} \PYG{o}{\PYGZhy{}} \PYG{n}{B}\PYG{o}{/}\PYG{l+m+mi}{2}\PYG{p}{)}\PYG{p}{)}    
    \PYG{n}{x1} \PYG{o}{=} \PYG{n}{np}\PYG{o}{.}\PYG{n}{append}\PYG{p}{(}\PYG{n}{low1}\PYG{p}{,} \PYG{n}{high}\PYG{p}{)}
    \PYG{n}{rect} \PYG{o}{=} \PYG{n}{np}\PYG{o}{.}\PYG{n}{append}\PYG{p}{(}\PYG{n}{x1}\PYG{p}{,} \PYG{n}{low1}\PYG{p}{)}
    
    \PYG{k}{if} \PYG{n}{x} \PYG{o}{\PYGZgt{}} \PYG{n+nb}{len}\PYG{p}{(}\PYG{n}{rect}\PYG{p}{)}\PYG{p}{:}
        \PYG{n}{rect} \PYG{o}{=} \PYG{n}{np}\PYG{o}{.}\PYG{n}{append}\PYG{p}{(}\PYG{n}{rect}\PYG{p}{,} \PYG{l+m+mi}{0}\PYG{p}{)}
    \PYG{k}{elif} \PYG{n}{x} \PYG{o}{\PYGZlt{}} \PYG{n+nb}{len}\PYG{p}{(}\PYG{n}{rect}\PYG{p}{)}\PYG{p}{:}
        \PYG{n}{rect} \PYG{o}{=} \PYG{n}{rect}\PYG{p}{[}\PYG{p}{:}\PYG{o}{\PYGZhy{}}\PYG{l+m+mi}{1}\PYG{p}{]}

    \PYG{k}{return} \PYG{n}{rect}
\end{sphinxVerbatim}

\end{sphinxuseclass}\end{sphinxVerbatimInput}

\end{sphinxuseclass}
\begin{sphinxuseclass}{cell}\begin{sphinxVerbatimInput}

\begin{sphinxuseclass}{cell_input}
\begin{sphinxVerbatim}[commandchars=\\\{\}]
\PYG{k+kn}{import} \PYG{n+nn}{numpy} \PYG{k}{as} \PYG{n+nn}{np}

\PYG{k}{def} \PYG{n+nf}{rect}\PYG{p}{(}\PYG{n}{x}\PYG{p}{,} \PYG{n}{B}\PYG{p}{)}\PYG{p}{:}
\PYG{+w}{    }\PYG{l+s+sd}{\PYGZdq{}\PYGZdq{}\PYGZdq{}}
\PYG{l+s+sd}{    create a rectangle function}
\PYG{l+s+sd}{    returns a numpy array that is 1 if |x| \PYGZlt{} w and 0 if |x| \PYGZgt{} w}
\PYG{l+s+sd}{    B is the rectangle width centered at 0}
\PYG{l+s+sd}{    x is the number of points in the array}
\PYG{l+s+sd}{    \PYGZdq{}\PYGZdq{}\PYGZdq{}}
    
    \PYG{n}{B} \PYG{o}{=} \PYG{n+nb}{int}\PYG{p}{(}\PYG{n}{B}\PYG{p}{)}
    \PYG{n}{x} \PYG{o}{=} \PYG{n+nb}{int}\PYG{p}{(}\PYG{n}{x}\PYG{p}{)}
    
    \PYG{n}{high} \PYG{o}{=} \PYG{n}{np}\PYG{o}{.}\PYG{n}{ones}\PYG{p}{(}\PYG{n}{B}\PYG{p}{)}
    \PYG{n}{low1} \PYG{o}{=} \PYG{n}{np}\PYG{o}{.}\PYG{n}{zeros}\PYG{p}{(}\PYG{n+nb}{int}\PYG{p}{(}\PYG{n}{x}\PYG{o}{/}\PYG{l+m+mi}{2} \PYG{o}{\PYGZhy{}} \PYG{n}{B}\PYG{o}{/}\PYG{l+m+mi}{2}\PYG{p}{)}\PYG{p}{)}    
    \PYG{n}{x1} \PYG{o}{=} \PYG{n}{np}\PYG{o}{.}\PYG{n}{append}\PYG{p}{(}\PYG{n}{low1}\PYG{p}{,} \PYG{n}{high}\PYG{p}{)}
    \PYG{n}{rect} \PYG{o}{=} \PYG{n}{np}\PYG{o}{.}\PYG{n}{append}\PYG{p}{(}\PYG{n}{x1}\PYG{p}{,} \PYG{n}{low1}\PYG{p}{)}
    
    \PYG{k}{if} \PYG{n}{x} \PYG{o}{\PYGZgt{}} \PYG{n+nb}{len}\PYG{p}{(}\PYG{n}{rect}\PYG{p}{)}\PYG{p}{:}
        \PYG{n}{rect} \PYG{o}{=} \PYG{n}{np}\PYG{o}{.}\PYG{n}{append}\PYG{p}{(}\PYG{n}{rect}\PYG{p}{,} \PYG{l+m+mi}{0}\PYG{p}{)}
    \PYG{k}{elif} \PYG{n}{x} \PYG{o}{\PYGZlt{}} \PYG{n+nb}{len}\PYG{p}{(}\PYG{n}{rect}\PYG{p}{)}\PYG{p}{:}
        \PYG{n}{rect} \PYG{o}{=} \PYG{n}{rect}\PYG{p}{[}\PYG{p}{:}\PYG{o}{\PYGZhy{}}\PYG{l+m+mi}{1}\PYG{p}{]}

    \PYG{k}{return} \PYG{n}{rect}
\end{sphinxVerbatim}

\end{sphinxuseclass}\end{sphinxVerbatimInput}

\end{sphinxuseclass}
\begin{sphinxuseclass}{cell}\begin{sphinxVerbatimInput}

\begin{sphinxuseclass}{cell_input}
\begin{sphinxVerbatim}[commandchars=\\\{\}]
\PYG{o}{\PYGZpc{}}\PYG{k}{matplotlib} notebook
\PYG{c+c1}{\PYGZsh{} \PYGZpc{}matplotlib inline}
\PYG{k+kn}{import} \PYG{n+nn}{matplotlib}\PYG{n+nn}{.}\PYG{n+nn}{pyplot} \PYG{k}{as} \PYG{n+nn}{plt}
\PYG{k+kn}{from} \PYG{n+nn}{matplotlib}\PYG{n+nn}{.}\PYG{n+nn}{animation} \PYG{k+kn}{import} \PYG{n}{FuncAnimation}

\PYG{c+c1}{\PYGZsh{} constants and x array}
\PYG{n}{pi} \PYG{o}{=} \PYG{n}{np}\PYG{o}{.}\PYG{n}{pi}
\PYG{n}{length} \PYG{o}{=} \PYG{l+m+mi}{2000}
\PYG{n}{x} \PYG{o}{=} \PYG{n}{np}\PYG{o}{.}\PYG{n}{linspace}\PYG{p}{(}\PYG{o}{\PYGZhy{}}\PYG{l+m+mi}{1}\PYG{p}{,} \PYG{l+m+mi}{1}\PYG{p}{,} \PYG{n}{length}\PYG{p}{)}

\PYG{c+c1}{\PYGZsh{} create figure and axes }
\PYG{n}{fig}\PYG{p}{,} \PYG{p}{(}\PYG{n}{ax1}\PYG{p}{,} \PYG{n}{ax2}\PYG{p}{)} \PYG{o}{=} \PYG{n}{plt}\PYG{o}{.}\PYG{n}{subplots}\PYG{p}{(}\PYG{l+m+mi}{2}\PYG{p}{,} \PYG{n}{figsize}\PYG{o}{=}\PYG{p}{(}\PYG{l+m+mi}{12}\PYG{p}{,} \PYG{l+m+mi}{6}\PYG{p}{)}\PYG{p}{)}

\PYG{c+c1}{\PYGZsh{} creating our line objects for the plots}
\PYG{n}{sinc}\PYG{p}{,} \PYG{o}{=} \PYG{n}{ax1}\PYG{o}{.}\PYG{n}{plot}\PYG{p}{(}\PYG{n}{x}\PYG{p}{,} \PYG{n}{np}\PYG{o}{.}\PYG{n}{sin}\PYG{p}{(}\PYG{n}{x}\PYG{p}{)}\PYG{p}{,} \PYG{l+s+s1}{\PYGZsq{}}\PYG{l+s+s1}{\PYGZhy{}b}\PYG{l+s+s1}{\PYGZsq{}}\PYG{p}{)}
\PYG{n}{box}\PYG{p}{,} \PYG{o}{=} \PYG{n}{ax2}\PYG{o}{.}\PYG{n}{plot}\PYG{p}{(}\PYG{n}{x}\PYG{p}{,} \PYG{n}{np}\PYG{o}{.}\PYG{n}{sin}\PYG{p}{(}\PYG{n}{x}\PYG{p}{)}\PYG{p}{,} \PYG{l+s+s1}{\PYGZsq{}}\PYG{l+s+s1}{\PYGZhy{}r}\PYG{l+s+s1}{\PYGZsq{}}\PYG{p}{)}

\PYG{k}{def} \PYG{n+nf}{animate}\PYG{p}{(}\PYG{n}{B}\PYG{p}{)}\PYG{p}{:}
\PYG{+w}{    }\PYG{l+s+sd}{\PYGZdq{}\PYGZdq{}\PYGZdq{}}
\PYG{l+s+sd}{    this function gets called by FuncAnimation}
\PYG{l+s+sd}{    each time called, it will replot with a different width \PYGZdq{}B\PYGZdq{}}
\PYG{l+s+sd}{    }
\PYG{l+s+sd}{    B: rect width}
\PYG{l+s+sd}{    }
\PYG{l+s+sd}{    return:}
\PYG{l+s+sd}{        sinc: ydata}
\PYG{l+s+sd}{        box: ydata}
\PYG{l+s+sd}{    \PYGZdq{}\PYGZdq{}\PYGZdq{}}
    
    \PYG{c+c1}{\PYGZsh{} create our rect object}
    \PYG{n}{f} \PYG{o}{=} \PYG{n}{rect}\PYG{p}{(}\PYG{n+nb}{len}\PYG{p}{(}\PYG{n}{x}\PYG{p}{)}\PYG{p}{,} \PYG{n}{B}\PYG{p}{)}
    \PYG{n}{box}\PYG{o}{.}\PYG{n}{set\PYGZus{}ydata}\PYG{p}{(}\PYG{n}{f}\PYG{p}{)}
    
    \PYG{c+c1}{\PYGZsh{} create our sinc object}
    \PYG{n}{F} \PYG{o}{=} \PYG{p}{(}\PYG{n}{B} \PYG{o}{/} \PYG{n}{length}\PYG{p}{)} \PYG{o}{*} \PYG{n}{np}\PYG{o}{.}\PYG{n}{sin}\PYG{p}{(}\PYG{n}{x} \PYG{o}{*} \PYG{n}{B} \PYG{o}{/} \PYG{l+m+mi}{2}\PYG{p}{)} \PYG{o}{/} \PYG{p}{(}\PYG{n}{x} \PYG{o}{*} \PYG{n}{B} \PYG{o}{/} \PYG{l+m+mi}{2}\PYG{p}{)}
    \PYG{n}{sinc}\PYG{o}{.}\PYG{n}{set\PYGZus{}ydata}\PYG{p}{(}\PYG{n}{F}\PYG{p}{)}
    
    \PYG{c+c1}{\PYGZsh{} adjust the sinc plot height in a loop}
    \PYG{n}{ax1}\PYG{o}{.}\PYG{n}{set\PYGZus{}ylim}\PYG{p}{(}\PYG{n}{np}\PYG{o}{.}\PYG{n}{min}\PYG{p}{(}\PYG{n}{F}\PYG{p}{)}\PYG{p}{,} \PYG{n}{np}\PYG{o}{.}\PYG{n}{max}\PYG{p}{(}\PYG{n}{F}\PYG{p}{)}\PYG{p}{)}
    
    \PYG{c+c1}{\PYGZsh{} format the ax1 yticks}
    \PYG{n}{plt}\PYG{o}{.}\PYG{n}{setp}\PYG{p}{(}\PYG{n}{ax1}\PYG{p}{,} \PYG{n}{xticks}\PYG{o}{=}\PYG{p}{[}\PYG{o}{\PYGZhy{}}\PYG{l+m+mf}{0.25}\PYG{p}{,} \PYG{l+m+mf}{0.25}\PYG{p}{]}\PYG{p}{,} \PYG{n}{xticklabels}\PYG{o}{=}\PYG{p}{[}\PYG{l+s+s1}{\PYGZsq{}}\PYG{l+s+s1}{\PYGZhy{}1/4}\PYG{l+s+s1}{\PYGZsq{}}\PYG{p}{,} \PYG{l+s+s1}{\PYGZsq{}}\PYG{l+s+s1}{1/4}\PYG{l+s+s1}{\PYGZsq{}}\PYG{p}{]}\PYG{p}{,}
             \PYG{n}{yticks}\PYG{o}{=}\PYG{p}{[}\PYG{l+m+mi}{0}\PYG{p}{,} \PYG{n}{np}\PYG{o}{.}\PYG{n}{max}\PYG{p}{(}\PYG{n}{F}\PYG{p}{)}\PYG{p}{]}\PYG{p}{,} \PYG{n}{yticklabels}\PYG{o}{=}\PYG{p}{[}\PYG{l+s+s1}{\PYGZsq{}}\PYG{l+s+s1}{0}\PYG{l+s+s1}{\PYGZsq{}}\PYG{p}{,} \PYG{l+s+s1}{\PYGZsq{}}\PYG{l+s+s1}{B=}\PYG{l+s+si}{\PYGZob{}:.2f\PYGZcb{}}\PYG{l+s+s1}{\PYGZsq{}}\PYG{o}{.}\PYG{n}{format}\PYG{p}{(}\PYG{p}{(}\PYG{n}{B} \PYG{o}{/} \PYG{n}{length}\PYG{p}{)}\PYG{p}{)}\PYG{p}{]}\PYG{p}{)}
    
    \PYG{c+c1}{\PYGZsh{} format the ax2 xticks to move with the box}
    \PYG{n}{plt}\PYG{o}{.}\PYG{n}{setp}\PYG{p}{(}\PYG{n}{ax2}\PYG{p}{,} \PYG{n}{yticks}\PYG{o}{=}\PYG{p}{[}\PYG{l+m+mi}{0}\PYG{p}{,} \PYG{l+m+mi}{1}\PYG{p}{]}\PYG{p}{,} 
             \PYG{n}{xticks}\PYG{o}{=}\PYG{p}{[}\PYG{o}{\PYGZhy{}}\PYG{l+m+mi}{1}\PYG{p}{,} \PYG{o}{\PYGZhy{}}\PYG{l+m+mi}{1} \PYG{o}{*} \PYG{n}{B} \PYG{o}{/} \PYG{n}{length}\PYG{p}{,} \PYG{l+m+mi}{1} \PYG{o}{*} \PYG{n}{B} \PYG{o}{/} \PYG{n}{length}\PYG{p}{,} \PYG{l+m+mi}{1}\PYG{p}{]}\PYG{p}{,} \PYG{n}{xticklabels}\PYG{o}{=}\PYG{p}{[}\PYG{l+s+s1}{\PYGZsq{}}\PYG{l+s+s1}{\PYGZhy{}1}\PYG{l+s+s1}{\PYGZsq{}}\PYG{p}{,} \PYG{l+s+s1}{\PYGZsq{}}\PYG{l+s+s1}{\PYGZhy{}B/2}\PYG{l+s+s1}{\PYGZsq{}}\PYG{p}{,} \PYG{l+s+s1}{\PYGZsq{}}\PYG{l+s+s1}{B/2}\PYG{l+s+s1}{\PYGZsq{}}\PYG{p}{,} \PYG{l+s+s1}{\PYGZsq{}}\PYG{l+s+s1}{1}\PYG{l+s+s1}{\PYGZsq{}}\PYG{p}{]}\PYG{p}{)}
    
\PYG{k}{def} \PYG{n+nf}{init}\PYG{p}{(}\PYG{p}{)}\PYG{p}{:}
\PYG{+w}{    }\PYG{l+s+sd}{\PYGZdq{}\PYGZdq{}\PYGZdq{}}
\PYG{l+s+sd}{    initialize the figure}
\PYG{l+s+sd}{    \PYGZdq{}\PYGZdq{}\PYGZdq{}}
    
    \PYG{n}{ax2}\PYG{o}{.}\PYG{n}{set\PYGZus{}ylim}\PYG{p}{(}\PYG{o}{\PYGZhy{}}\PYG{l+m+mf}{0.2}\PYG{p}{,} \PYG{l+m+mf}{1.1}\PYG{p}{)}
    \PYG{n}{ax1}\PYG{o}{.}\PYG{n}{set\PYGZus{}xlim}\PYG{p}{(}\PYG{o}{\PYGZhy{}}\PYG{l+m+mf}{0.25}\PYG{p}{,} \PYG{l+m+mf}{0.25}\PYG{p}{)}
    \PYG{n}{ax2}\PYG{o}{.}\PYG{n}{set\PYGZus{}xlim}\PYG{p}{(}\PYG{o}{\PYGZhy{}}\PYG{l+m+mi}{1}\PYG{p}{,} \PYG{l+m+mi}{1}\PYG{p}{)}
    \PYG{n}{ax1}\PYG{o}{.}\PYG{n}{axhline}\PYG{p}{(}\PYG{l+m+mi}{0}\PYG{p}{,} \PYG{n}{color}\PYG{o}{=}\PYG{l+s+s1}{\PYGZsq{}}\PYG{l+s+s1}{black}\PYG{l+s+s1}{\PYGZsq{}}\PYG{p}{,} \PYG{n}{lw}\PYG{o}{=}\PYG{l+m+mi}{1}\PYG{p}{)}
    \PYG{n}{ax2}\PYG{o}{.}\PYG{n}{axhline}\PYG{p}{(}\PYG{l+m+mi}{0}\PYG{p}{,} \PYG{n}{color}\PYG{o}{=}\PYG{l+s+s1}{\PYGZsq{}}\PYG{l+s+s1}{black}\PYG{l+s+s1}{\PYGZsq{}}\PYG{p}{,} \PYG{n}{lw}\PYG{o}{=}\PYG{l+m+mi}{1}\PYG{p}{)}
    \PYG{n}{plt}\PYG{o}{.}\PYG{n}{rcParams}\PYG{o}{.}\PYG{n}{update}\PYG{p}{(}\PYG{p}{\PYGZob{}}\PYG{l+s+s1}{\PYGZsq{}}\PYG{l+s+s1}{font.size}\PYG{l+s+s1}{\PYGZsq{}}\PYG{p}{:}\PYG{l+m+mi}{14}\PYG{p}{\PYGZcb{}}\PYG{p}{)}
    
    \PYG{k}{return} \PYG{n}{sinc}\PYG{p}{,} \PYG{n}{box}\PYG{p}{,}

\PYG{c+c1}{\PYGZsh{} the FuncAnimation function iterates through our animate function using the steps array}
\PYG{n}{step} \PYG{o}{=} \PYG{l+m+mi}{10}
\PYG{n}{steps} \PYG{o}{=} \PYG{n}{np}\PYG{o}{.}\PYG{n}{append}\PYG{p}{(}\PYG{n}{np}\PYG{o}{.}\PYG{n}{arange}\PYG{p}{(}\PYG{l+m+mi}{10}\PYG{p}{,} \PYG{l+m+mi}{1000}\PYG{p}{,} \PYG{n}{step}\PYG{p}{)}\PYG{p}{,} \PYG{n}{np}\PYG{o}{.}\PYG{n}{arange}\PYG{p}{(}\PYG{l+m+mi}{1000}\PYG{p}{,} \PYG{l+m+mi}{10}\PYG{p}{,} \PYG{o}{\PYGZhy{}}\PYG{l+m+mi}{1} \PYG{o}{*} \PYG{n}{step}\PYG{p}{)}\PYG{p}{)}
\PYG{n}{ani} \PYG{o}{=} \PYG{n}{FuncAnimation}\PYG{p}{(}\PYG{n}{fig}\PYG{p}{,} \PYG{n}{animate}\PYG{p}{,} \PYG{n}{steps}\PYG{p}{,} \PYG{n}{init\PYGZus{}func}\PYG{o}{=}\PYG{n}{init}\PYG{p}{,} \PYG{n}{interval}\PYG{o}{=}\PYG{l+m+mi}{50}\PYG{p}{,} \PYG{n}{blit}\PYG{o}{=}\PYG{k+kc}{True}\PYG{p}{)}
\PYG{n}{plt}\PYG{o}{.}\PYG{n}{show}\PYG{p}{(}\PYG{p}{)}
\end{sphinxVerbatim}

\end{sphinxuseclass}\end{sphinxVerbatimInput}
\begin{sphinxVerbatimOutput}

\begin{sphinxuseclass}{cell_output}
\begin{sphinxVerbatim}[commandchars=\\\{\}]
\PYGZlt{}IPython.core.display.Javascript object\PYGZgt{}
\end{sphinxVerbatim}

\begin{sphinxVerbatim}[commandchars=\\\{\}]
\PYGZlt{}IPython.core.display.HTML object\PYGZgt{}
\end{sphinxVerbatim}

\end{sphinxuseclass}\end{sphinxVerbatimOutput}

\end{sphinxuseclass}
\begin{sphinxuseclass}{cell}\begin{sphinxVerbatimInput}

\begin{sphinxuseclass}{cell_input}
\begin{sphinxVerbatim}[commandchars=\\\{\}]
\PYG{k+kn}{import} \PYG{n+nn}{numpy} \PYG{k}{as} \PYG{n+nn}{np}

\PYG{k}{def} \PYG{n+nf}{rect}\PYG{p}{(}\PYG{n}{x}\PYG{p}{,} \PYG{n}{B}\PYG{p}{)}\PYG{p}{:}
\PYG{+w}{    }\PYG{l+s+sd}{\PYGZdq{}\PYGZdq{}\PYGZdq{}}
\PYG{l+s+sd}{    create a rectangle function}
\PYG{l+s+sd}{    returns a numpy array that is 1 if |x| \PYGZlt{} w and 0 if |x| \PYGZgt{} w}
\PYG{l+s+sd}{    B is the rectangle width centered at 0}
\PYG{l+s+sd}{    x is the number of points in the array}
\PYG{l+s+sd}{    \PYGZdq{}\PYGZdq{}\PYGZdq{}}
    
    \PYG{n}{B} \PYG{o}{=} \PYG{n+nb}{int}\PYG{p}{(}\PYG{n}{B}\PYG{p}{)}
    \PYG{n}{x} \PYG{o}{=} \PYG{n+nb}{int}\PYG{p}{(}\PYG{n}{x}\PYG{p}{)}
    
    \PYG{n}{high} \PYG{o}{=} \PYG{n}{np}\PYG{o}{.}\PYG{n}{ones}\PYG{p}{(}\PYG{n}{B}\PYG{p}{)}
    \PYG{n}{low1} \PYG{o}{=} \PYG{n}{np}\PYG{o}{.}\PYG{n}{zeros}\PYG{p}{(}\PYG{n+nb}{int}\PYG{p}{(}\PYG{n}{x}\PYG{o}{/}\PYG{l+m+mi}{2} \PYG{o}{\PYGZhy{}} \PYG{n}{B}\PYG{o}{/}\PYG{l+m+mi}{2}\PYG{p}{)}\PYG{p}{)}    
    \PYG{n}{x1} \PYG{o}{=} \PYG{n}{np}\PYG{o}{.}\PYG{n}{append}\PYG{p}{(}\PYG{n}{low1}\PYG{p}{,} \PYG{n}{high}\PYG{p}{)}
    \PYG{n}{rect} \PYG{o}{=} \PYG{n}{np}\PYG{o}{.}\PYG{n}{append}\PYG{p}{(}\PYG{n}{x1}\PYG{p}{,} \PYG{n}{low1}\PYG{p}{)}
    
    \PYG{k}{if} \PYG{n}{x} \PYG{o}{\PYGZgt{}} \PYG{n+nb}{len}\PYG{p}{(}\PYG{n}{rect}\PYG{p}{)}\PYG{p}{:}
        \PYG{n}{rect} \PYG{o}{=} \PYG{n}{np}\PYG{o}{.}\PYG{n}{append}\PYG{p}{(}\PYG{n}{rect}\PYG{p}{,} \PYG{l+m+mi}{0}\PYG{p}{)}
    \PYG{k}{elif} \PYG{n}{x} \PYG{o}{\PYGZlt{}} \PYG{n+nb}{len}\PYG{p}{(}\PYG{n}{rect}\PYG{p}{)}\PYG{p}{:}
        \PYG{n}{rect} \PYG{o}{=} \PYG{n}{rect}\PYG{p}{[}\PYG{p}{:}\PYG{o}{\PYGZhy{}}\PYG{l+m+mi}{1}\PYG{p}{]}

    \PYG{k}{return} \PYG{n}{rect}
\end{sphinxVerbatim}

\end{sphinxuseclass}\end{sphinxVerbatimInput}

\end{sphinxuseclass}
\sphinxstepscope


\chapter{testing jupyter\sphinxhyphen{}book ok using fFy examples}
\label{\detokenize{fFy:testing-jupyter-book-ok-using-ffy-examples}}\label{\detokenize{fFy::doc}}
\sphinxAtStartPar
from \sphinxurl{https://github.com/markjay4k/fourier-transform}
as only use matplotlib inline

\sphinxAtStartPar
But there is no plot just wedge …


\section{Fourier Transform}
\label{\detokenize{fFy:fourier-transform}}\begin{enumerate}
\sphinxsetlistlabels{\arabic}{enumi}{enumii}{}{.}%
\item {} 
\sphinxAtStartPar
Fourier Transform is a generalized version of the Fourier Series

\item {} 
\sphinxAtStartPar
It applies to both period and non periodic functions
\begin{itemize}
\item {} 
\sphinxAtStartPar
For periodic functions, the spectrum is discrete

\item {} 
\sphinxAtStartPar
For non\sphinxhyphen{}period functions, the spectrum is continuous

\end{itemize}

\end{enumerate}


\subsection{Definitions}
\label{\detokenize{fFy:definitions}}

\subsubsection{Fourier Transform}
\label{\detokenize{fFy:id1}}
\sphinxAtStartPar
Fourier Transform of \(f(x)\) is \(F(k)\)
\$\(
F(k) = \mathcal{FT}\{f(x)\}
\)\$
\begin{equation*}
\begin{split}
F(k) = \int_{-\infty}^{\infty}f(x) \exp(-ikx)dx
\end{split}
\end{equation*}
\sphinxAtStartPar
where \(k=\frac{2\pi}{x}\) is called the “wavenumber”


\subsubsection{Inverse Fourier Transform}
\label{\detokenize{fFy:inverse-fourier-transform}}
\sphinxAtStartPar
To go back to \(f(x)\), the formula is
\begin{equation*}
\begin{split}
f(x) = \mathcal{FT}^{-1}\{F(k)\}
\end{split}
\end{equation*}\begin{equation*}
\begin{split}
f(x) = \frac{1}{2\pi} \int_{-\infty}^{\infty}F(k) \exp(ikx)dx
\end{split}
\end{equation*}
\sphinxAtStartPar
Since \(x\) and \(k\) are inversely proportional, the “size” of \(f(x)\) and \(F(k)\) are inversely proportional.
What this means is,
\begin{itemize}
\item {} 
\sphinxAtStartPar
a compact \(f(x)\) will have a broad spectrum.

\item {} 
\sphinxAtStartPar
a broad \(f(x)\) will have a compact spectrum

\end{itemize}


\subsection{Rectangle function}
\label{\detokenize{fFy:rectangle-function}}
\sphinxAtStartPar
The \(rext_B(x)\) function is a rectangle centered at \(x=0\) with \(\text{Height}=1\) and \(\text{Width}=B\). The Formula can be written as
\begin{equation*}
\begin{split}
\text{rect}_B(x) =
\begin{cases}
0 & \text{if} \,\,\, |x| > B/2\\[2pt]
1/2 & \text{if} \,\,\, |x| = B/2 \\[2pt]
1 & \text{if} \,\,\, |x| < B/2
\end{cases}
\end{split}
\end{equation*}
\sphinxAtStartPar
The cell below is a simple function for creating \(\text{rect}_B(x)\)


\subsection{Example Fourier Transform of \protect\(\text{rect}\protect\) function}
\label{\detokenize{fFy:example-fourier-transform-of-text-rect-function}}
\sphinxAtStartPar
Using the FT definition and the \(\text{rect}_B(x)\) equation, the FT is
\begin{equation*}
\begin{split}
F(k) = \int_{-B/2}^{B/2} \exp(-ikx)dx
\end{split}
\end{equation*}\begin{equation*}
\begin{split}
= -\frac{1}{ik} \exp(-ikx) \, \Big|_{-B/2}^{\,B/2}
\end{split}
\end{equation*}\begin{equation*}
\begin{split}
= -\frac{1}{ik} \left[ \exp(-ikB/2) - \exp(ikB/2)\right]
\end{split}
\end{equation*}
\sphinxAtStartPar
Using the complex definition of sine from Euler’s formula
\begin{equation*}
\begin{split}
\sin(x) = \frac{e^{ix} - e^{-ix}}{2i}
\end{split}
\end{equation*}
\sphinxAtStartPar
Our equation for \(F(k)\) can be re\sphinxhyphen{}written as
\begin{equation*}
\begin{split}
F(k) = \frac{2}{k}\frac{\exp(ikB/2) - \exp(-ikB/2)}{2i}
\end{split}
\end{equation*}\begin{equation*}
\begin{split}
= \frac{2}{k} \sin(kB/2)
\end{split}
\end{equation*}\begin{equation*}
\begin{split}
= B \frac{\sin(kB/2)}{kB/2}
\end{split}
\end{equation*}\begin{equation*}
\begin{split}
F(k) = B \text{sinc}(kB/2)
\end{split}
\end{equation*}
\begin{sphinxuseclass}{cell}\begin{sphinxVerbatimInput}

\begin{sphinxuseclass}{cell_input}
\begin{sphinxVerbatim}[commandchars=\\\{\}]
\PYG{k+kn}{import} \PYG{n+nn}{numpy} \PYG{k}{as} \PYG{n+nn}{np}

\PYG{k}{def} \PYG{n+nf}{rect}\PYG{p}{(}\PYG{n}{x}\PYG{p}{,} \PYG{n}{B}\PYG{p}{)}\PYG{p}{:}
\PYG{+w}{    }\PYG{l+s+sd}{\PYGZdq{}\PYGZdq{}\PYGZdq{}}
\PYG{l+s+sd}{    create a rectangle function}
\PYG{l+s+sd}{    returns a numpy array that is 1 if |x| \PYGZlt{} w and 0 if |x| \PYGZgt{} w}
\PYG{l+s+sd}{    w is the rectangle width centered at 0}
\PYG{l+s+sd}{    x is the number of points in the array}
\PYG{l+s+sd}{    \PYGZdq{}\PYGZdq{}\PYGZdq{}}
    
    \PYG{n}{B} \PYG{o}{=} \PYG{n+nb}{int}\PYG{p}{(}\PYG{n}{B}\PYG{p}{)}
    \PYG{n}{x} \PYG{o}{=} \PYG{n+nb}{int}\PYG{p}{(}\PYG{n}{x}\PYG{p}{)}
    
    \PYG{n}{high} \PYG{o}{=} \PYG{n}{np}\PYG{o}{.}\PYG{n}{ones}\PYG{p}{(}\PYG{n}{B}\PYG{p}{)}
    \PYG{n}{low1} \PYG{o}{=} \PYG{n}{np}\PYG{o}{.}\PYG{n}{zeros}\PYG{p}{(}\PYG{n+nb}{int}\PYG{p}{(}\PYG{n}{x}\PYG{o}{/}\PYG{l+m+mi}{2} \PYG{o}{\PYGZhy{}} \PYG{n}{B}\PYG{o}{/}\PYG{l+m+mi}{2}\PYG{p}{)}\PYG{p}{)}    
    \PYG{n}{x1} \PYG{o}{=} \PYG{n}{np}\PYG{o}{.}\PYG{n}{append}\PYG{p}{(}\PYG{n}{low1}\PYG{p}{,} \PYG{n}{high}\PYG{p}{)}
    \PYG{n}{rect} \PYG{o}{=} \PYG{n}{np}\PYG{o}{.}\PYG{n}{append}\PYG{p}{(}\PYG{n}{x1}\PYG{p}{,} \PYG{n}{low1}\PYG{p}{)}
    
    \PYG{k}{if} \PYG{n}{x} \PYG{o}{\PYGZgt{}} \PYG{n+nb}{len}\PYG{p}{(}\PYG{n}{rect}\PYG{p}{)}\PYG{p}{:}
        \PYG{n}{rect} \PYG{o}{=} \PYG{n}{np}\PYG{o}{.}\PYG{n}{append}\PYG{p}{(}\PYG{n}{rect}\PYG{p}{,} \PYG{l+m+mi}{0}\PYG{p}{)}
    \PYG{k}{elif} \PYG{n}{x} \PYG{o}{\PYGZlt{}} \PYG{n+nb}{len}\PYG{p}{(}\PYG{n}{rect}\PYG{p}{)}\PYG{p}{:}
        \PYG{n}{rect} \PYG{o}{=} \PYG{n}{rect}\PYG{p}{[}\PYG{p}{:}\PYG{o}{\PYGZhy{}}\PYG{l+m+mi}{1}\PYG{p}{]}

    \PYG{k}{return} \PYG{n}{rect}
\end{sphinxVerbatim}

\end{sphinxuseclass}\end{sphinxVerbatimInput}

\end{sphinxuseclass}
\begin{sphinxuseclass}{cell}\begin{sphinxVerbatimInput}

\begin{sphinxuseclass}{cell_input}
\begin{sphinxVerbatim}[commandchars=\\\{\}]
\PYG{o}{\PYGZpc{}}\PYG{k}{matplotlib} inline
\PYG{o}{\PYGZpc{}}\PYG{k}{config} InlineBackend.figure\PYGZus{}format = \PYGZsq{}svg\PYGZsq{}
\PYG{k+kn}{import} \PYG{n+nn}{matplotlib}\PYG{n+nn}{.}\PYG{n+nn}{pyplot} \PYG{k}{as} \PYG{n+nn}{plt}
\PYG{k+kn}{from} \PYG{n+nn}{IPython} \PYG{k+kn}{import} \PYG{n}{display} \PYG{k}{as} \PYG{n}{disp}

\PYG{k+kn}{import} \PYG{n+nn}{ipywidgets} \PYG{k}{as} \PYG{n+nn}{widgets}
\PYG{k+kn}{from} \PYG{n+nn}{IPython}\PYG{n+nn}{.}\PYG{n+nn}{display} \PYG{k+kn}{import} \PYG{n}{display}
\PYG{n}{slide} \PYG{o}{=} \PYG{n}{widgets}\PYG{o}{.}\PYG{n}{IntSlider}\PYG{p}{(}\PYG{p}{)}
\PYG{n}{display}\PYG{p}{(}\PYG{n}{slide}\PYG{p}{)}

\PYG{k+kn}{from} \PYG{n+nn}{IPython}\PYG{n+nn}{.}\PYG{n+nn}{display} \PYG{k+kn}{import} \PYG{n}{display}
\PYG{n}{button} \PYG{o}{=} \PYG{n}{widgets}\PYG{o}{.}\PYG{n}{Button}\PYG{p}{(}\PYG{n}{description}\PYG{o}{=}\PYG{l+s+s2}{\PYGZdq{}}\PYG{l+s+s2}{update plot}\PYG{l+s+s2}{\PYGZdq{}}\PYG{p}{)}
\PYG{n}{display}\PYG{p}{(}\PYG{n}{button}\PYG{p}{)}

\PYG{n}{pi} \PYG{o}{=} \PYG{n}{np}\PYG{o}{.}\PYG{n}{pi}
\PYG{n}{length} \PYG{o}{=} \PYG{l+m+mi}{2000}
\PYG{n}{x} \PYG{o}{=} \PYG{n}{np}\PYG{o}{.}\PYG{n}{linspace}\PYG{p}{(}\PYG{o}{\PYGZhy{}}\PYG{l+m+mi}{1}\PYG{p}{,} \PYG{l+m+mi}{1}\PYG{p}{,} \PYG{n}{length}\PYG{p}{)}

\PYG{k}{def} \PYG{n+nf}{on\PYGZus{}button\PYGZus{}clicked}\PYG{p}{(}\PYG{n}{b}\PYG{p}{)}\PYG{p}{:}
\PYG{+w}{    }\PYG{l+s+sd}{\PYGZdq{}\PYGZdq{}\PYGZdq{}}
\PYG{l+s+sd}{    excecutes function when button is clicked}
\PYG{l+s+sd}{    \PYGZdq{}\PYGZdq{}\PYGZdq{}}
    \PYG{n}{B} \PYG{o}{=} \PYG{n}{slide}\PYG{o}{.}\PYG{n}{value} \PYG{o}{*} \PYG{l+m+mi}{10}
    \PYG{k}{if} \PYG{n}{B} \PYG{o}{==} \PYG{l+m+mi}{0}\PYG{p}{:}
        \PYG{n}{B} \PYG{o}{=} \PYG{l+m+mi}{10}
    \PYG{n}{plt}\PYG{o}{.}\PYG{n}{rcParams}\PYG{o}{.}\PYG{n}{update}\PYG{p}{(}\PYG{p}{\PYGZob{}}\PYG{l+s+s1}{\PYGZsq{}}\PYG{l+s+s1}{font.size}\PYG{l+s+s1}{\PYGZsq{}}\PYG{p}{:} \PYG{l+m+mi}{14}\PYG{p}{\PYGZcb{}}\PYG{p}{)}
    \PYG{n}{plt}\PYG{o}{.}\PYG{n}{rcParams}\PYG{p}{[}\PYG{l+s+s1}{\PYGZsq{}}\PYG{l+s+s1}{figure.figsize}\PYG{l+s+s1}{\PYGZsq{}}\PYG{p}{]} \PYG{o}{=} \PYG{p}{(}\PYG{l+m+mi}{12}\PYG{p}{,} \PYG{l+m+mf}{1.9}\PYG{p}{)}
    \PYG{n}{plt}\PYG{o}{.}\PYG{n}{yticks}\PYG{p}{(}\PYG{p}{[}\PYG{l+m+mi}{0}\PYG{p}{,} \PYG{l+m+mi}{1}\PYG{p}{]}\PYG{p}{,} \PYG{p}{[}\PYG{l+s+s1}{\PYGZsq{}}\PYG{l+s+s1}{\PYGZdl{}0\PYGZdl{}}\PYG{l+s+s1}{\PYGZsq{}}\PYG{p}{,} \PYG{l+s+s1}{\PYGZsq{}}\PYG{l+s+s1}{\PYGZdl{}1\PYGZdl{}}\PYG{l+s+s1}{\PYGZsq{}}\PYG{p}{]}\PYG{p}{)}
    \PYG{n}{plt}\PYG{o}{.}\PYG{n}{xticks}\PYG{p}{(}\PYG{p}{[}\PYG{o}{\PYGZhy{}}\PYG{l+m+mi}{1}\PYG{o}{*}\PYG{n}{B}\PYG{o}{/}\PYG{n}{length}\PYG{p}{,} \PYG{l+m+mi}{1}\PYG{o}{*}\PYG{n}{B}\PYG{o}{/}\PYG{n}{length}\PYG{p}{]}\PYG{p}{,} \PYG{p}{[}\PYG{l+s+s1}{\PYGZsq{}}\PYG{l+s+s1}{\PYGZdl{}\PYGZhy{}B/2\PYGZdl{}}\PYG{l+s+s1}{\PYGZsq{}}\PYG{p}{,} \PYG{l+s+s1}{\PYGZsq{}}\PYG{l+s+s1}{\PYGZdl{}B/2\PYGZdl{}}\PYG{l+s+s1}{\PYGZsq{}}\PYG{p}{]}\PYG{p}{)}
    \PYG{n}{plt}\PYG{o}{.}\PYG{n}{plot}\PYG{p}{(}\PYG{n}{x}\PYG{p}{,} \PYG{n}{rect}\PYG{p}{(}\PYG{n+nb}{len}\PYG{p}{(}\PYG{n}{x}\PYG{p}{)}\PYG{p}{,} \PYG{n}{B}\PYG{p}{)}\PYG{p}{,} \PYG{n}{label}\PYG{o}{=}\PYG{l+s+sa}{r}\PYG{l+s+s1}{\PYGZsq{}}\PYG{l+s+s1}{\PYGZdl{}f(x)=rect\PYGZus{}B(x)\PYGZdl{}}\PYG{l+s+s1}{\PYGZsq{}}\PYG{p}{)}
    \PYG{n}{plt}\PYG{o}{.}\PYG{n}{axhline}\PYG{p}{(}\PYG{l+m+mi}{0}\PYG{p}{,} \PYG{n}{color}\PYG{o}{=}\PYG{l+s+s1}{\PYGZsq{}}\PYG{l+s+s1}{black}\PYG{l+s+s1}{\PYGZsq{}}\PYG{p}{,} \PYG{n}{lw}\PYG{o}{=}\PYG{l+m+mi}{1}\PYG{p}{)}
    \PYG{n}{leg} \PYG{o}{=} \PYG{n}{plt}\PYG{o}{.}\PYG{n}{legend}\PYG{p}{(}\PYG{n}{loc}\PYG{o}{=}\PYG{l+s+s1}{\PYGZsq{}}\PYG{l+s+s1}{best}\PYG{l+s+s1}{\PYGZsq{}}\PYG{p}{,} \PYG{n}{fontsize}\PYG{o}{=}\PYG{l+m+mi}{14}\PYG{p}{,} \PYG{n}{fancybox}\PYG{o}{=}\PYG{k+kc}{True}\PYG{p}{)}
    \PYG{n}{leg}\PYG{o}{.}\PYG{n}{get\PYGZus{}frame}\PYG{p}{(}\PYG{p}{)}\PYG{o}{.}\PYG{n}{set\PYGZus{}linewidth}\PYG{p}{(}\PYG{l+m+mf}{0.1}\PYG{p}{)}
    \PYG{n}{plt}\PYG{o}{.}\PYG{n}{xlabel}\PYG{p}{(}\PYG{l+s+s1}{\PYGZsq{}}\PYG{l+s+s1}{\PYGZdl{}x\PYGZdl{}}\PYG{l+s+s1}{\PYGZsq{}}\PYG{p}{)}
    \PYG{n}{plt}\PYG{o}{.}\PYG{n}{ylim}\PYG{p}{(}\PYG{o}{\PYGZhy{}}\PYG{l+m+mf}{0.2}\PYG{p}{,} \PYG{l+m+mf}{1.2}\PYG{p}{)}
    \PYG{n}{plt}\PYG{o}{.}\PYG{n}{show}\PYG{p}{(}\PYG{p}{)}
    
    \PYG{n}{plt}\PYG{o}{.}\PYG{n}{yticks}\PYG{p}{(}\PYG{p}{[}\PYG{l+m+mi}{0}\PYG{p}{,} \PYG{l+m+mi}{1}\PYG{p}{]}\PYG{p}{,} \PYG{p}{[}\PYG{l+s+s1}{\PYGZsq{}}\PYG{l+s+s1}{\PYGZdl{}0\PYGZdl{}}\PYG{l+s+s1}{\PYGZsq{}}\PYG{p}{,} \PYG{l+s+s1}{\PYGZsq{}}\PYG{l+s+s1}{\PYGZdl{}1\PYGZdl{}}\PYG{l+s+s1}{\PYGZsq{}}\PYG{p}{]}\PYG{p}{)}
    \PYG{n}{plt}\PYG{o}{.}\PYG{n}{xticks}\PYG{p}{(}\PYG{p}{[}\PYG{o}{\PYGZhy{}}\PYG{l+m+mi}{1}\PYG{o}{*}\PYG{n}{pi}\PYG{p}{,} \PYG{l+m+mi}{0}\PYG{p}{,} \PYG{l+m+mi}{1}\PYG{o}{*}\PYG{n}{pi}\PYG{p}{]}\PYG{p}{,} \PYG{p}{[}\PYG{l+s+s1}{\PYGZsq{}}\PYG{l+s+s1}{\PYGZdl{}\PYGZhy{}B/2\PYGZdl{}}\PYG{l+s+s1}{\PYGZsq{}}\PYG{p}{,} \PYG{l+s+s1}{\PYGZsq{}}\PYG{l+s+s1}{\PYGZdl{}0\PYGZdl{}}\PYG{l+s+s1}{\PYGZsq{}}\PYG{p}{,} \PYG{l+s+s1}{\PYGZsq{}}\PYG{l+s+s1}{\PYGZdl{}B/2\PYGZdl{}}\PYG{l+s+s1}{\PYGZsq{}}\PYG{p}{]}\PYG{p}{)}

    \PYG{n}{k} \PYG{o}{=} \PYG{n}{np}\PYG{o}{.}\PYG{n}{linspace}\PYG{p}{(}\PYG{o}{\PYGZhy{}}\PYG{l+m+mi}{1}\PYG{p}{,} \PYG{l+m+mi}{1}\PYG{p}{,} \PYG{n}{length}\PYG{p}{)}
    \PYG{n}{plt}\PYG{o}{.}\PYG{n}{plot}\PYG{p}{(}\PYG{n}{x}\PYG{p}{,} \PYG{p}{(}\PYG{n}{B} \PYG{o}{/} \PYG{n}{length}\PYG{p}{)} \PYG{o}{*} \PYG{n}{np}\PYG{o}{.}\PYG{n}{sin}\PYG{p}{(}\PYG{n}{k} \PYG{o}{*} \PYG{n}{B} \PYG{o}{/} \PYG{l+m+mi}{2}\PYG{p}{)} \PYG{o}{/} \PYG{p}{(}\PYG{n}{B} \PYG{o}{*} \PYG{n}{k} \PYG{o}{/} \PYG{l+m+mi}{2}\PYG{p}{)}\PYG{p}{,} \PYG{l+s+s1}{\PYGZsq{}}\PYG{l+s+s1}{r}\PYG{l+s+s1}{\PYGZsq{}}\PYG{p}{,} \PYG{n}{label}\PYG{o}{=}\PYG{l+s+sa}{r}\PYG{l+s+s1}{\PYGZsq{}}\PYG{l+s+s1}{\PYGZdl{}F(k)=B sinc(kB/2)\PYGZdl{}}\PYG{l+s+s1}{\PYGZsq{}}\PYG{p}{)}
    \PYG{n}{plt}\PYG{o}{.}\PYG{n}{axhline}\PYG{p}{(}\PYG{l+m+mi}{0}\PYG{p}{,} \PYG{n}{color}\PYG{o}{=}\PYG{l+s+s1}{\PYGZsq{}}\PYG{l+s+s1}{black}\PYG{l+s+s1}{\PYGZsq{}}\PYG{p}{,} \PYG{n}{lw}\PYG{o}{=}\PYG{l+m+mi}{1}\PYG{p}{)}
    \PYG{n}{leg} \PYG{o}{=} \PYG{n}{plt}\PYG{o}{.}\PYG{n}{legend}\PYG{p}{(}\PYG{n}{loc}\PYG{o}{=}\PYG{l+s+s1}{\PYGZsq{}}\PYG{l+s+s1}{best}\PYG{l+s+s1}{\PYGZsq{}}\PYG{p}{,} \PYG{n}{fontsize}\PYG{o}{=}\PYG{l+m+mi}{14}\PYG{p}{,} \PYG{n}{fancybox}\PYG{o}{=}\PYG{k+kc}{True}\PYG{p}{)}
    \PYG{n}{leg}\PYG{o}{.}\PYG{n}{get\PYGZus{}frame}\PYG{p}{(}\PYG{p}{)}\PYG{o}{.}\PYG{n}{set\PYGZus{}linewidth}\PYG{p}{(}\PYG{l+m+mf}{0.1}\PYG{p}{)}
    \PYG{n}{plt}\PYG{o}{.}\PYG{n}{xlabel}\PYG{p}{(}\PYG{l+s+s1}{\PYGZsq{}}\PYG{l+s+s1}{\PYGZdl{}k\PYGZdl{}}\PYG{l+s+s1}{\PYGZsq{}}\PYG{p}{)}
    \PYG{n}{plt}\PYG{o}{.}\PYG{n}{xlim}\PYG{p}{(}\PYG{o}{\PYGZhy{}}\PYG{l+m+mf}{.25}\PYG{p}{,} \PYG{l+m+mf}{.25}\PYG{p}{)}
    \PYG{n}{plt}\PYG{o}{.}\PYG{n}{show}\PYG{p}{(}\PYG{p}{)}
    \PYG{n}{disp}\PYG{o}{.}\PYG{n}{clear\PYGZus{}output}\PYG{p}{(}\PYG{n}{wait}\PYG{o}{=}\PYG{k+kc}{True}\PYG{p}{)}
 
\PYG{n}{button}\PYG{o}{.}\PYG{n}{on\PYGZus{}click}\PYG{p}{(}\PYG{n}{on\PYGZus{}button\PYGZus{}clicked}\PYG{p}{)}
\end{sphinxVerbatim}

\end{sphinxuseclass}\end{sphinxVerbatimInput}
\begin{sphinxVerbatimOutput}

\begin{sphinxuseclass}{cell_output}
\begin{sphinxVerbatim}[commandchars=\\\{\}]
IntSlider(value=0)
\end{sphinxVerbatim}

\begin{sphinxVerbatim}[commandchars=\\\{\}]
Button(description=\PYGZsq{}update plot\PYGZsq{}, style=ButtonStyle())
\end{sphinxVerbatim}

\end{sphinxuseclass}\end{sphinxVerbatimOutput}

\end{sphinxuseclass}
\sphinxstepscope


\chapter{testing jupyter\sphinxhyphen{}book ok using fnn examples}
\label{\detokenize{fnn:testing-jupyter-book-ok-using-fnn-examples}}\label{\detokenize{fnn::doc}}
\sphinxAtStartPar
from \sphinxurl{https://github.com/markjay4k/fourier-transform}
as only use matplotlib inline

\sphinxAtStartPar
NUMERICAL INTEGRATION

\sphinxAtStartPar
What is numerical Integration

\sphinxAtStartPar
Numerical Integration is a way to approximate the value of a definite integral

\sphinxAtStartPar
Unlike Analytic solutions which are exact, closed form solutions

\sphinxAtStartPar
Numerical Integration
There are many options for numerically computing definite integrals:
\begin{itemize}
\item {} 
\sphinxAtStartPar
Trapezoid Rule

\item {} 
\sphinxAtStartPar
Simpsons Rule

\item {} 
\sphinxAtStartPar
Gaussian Quadrature

\item {} 
\sphinxAtStartPar
etc.

\end{itemize}

\sphinxAtStartPar
Let’s look at the Trapezoid Rule

\sphinxAtStartPar
The Trapezoid rule is a way of approximating a definite integral by breaking it up into trapezoids.

\sphinxAtStartPar
Let’s say we want to compute the integral
\begin{equation*}
\begin{split} \int_0^{\pi} \sin(x)dx \end{split}
\end{equation*}
\sphinxAtStartPar
We can easily compute it analytically
\begin{equation*}
\begin{split} \int_0^{\pi} \sin(x)dx = -\cos(x) \big|_0^{\pi} \end{split}
\end{equation*}\begin{equation*}
\begin{split} =-(\cos(\pi) - \cos(0)) \end{split}
\end{equation*}\begin{equation*}
\begin{split} = -(-1 - 1)\end{split}
\end{equation*}\begin{equation*}
\begin{split} 2 \end{split}
\end{equation*}
\sphinxAtStartPar
But let’s try computing using the trapezoid rule.

\sphinxAtStartPar
Below is a visualization showing \(\sin(x)\) and a trapezoid approximation.

\begin{sphinxuseclass}{cell}\begin{sphinxVerbatimInput}

\begin{sphinxuseclass}{cell_input}
\begin{sphinxVerbatim}[commandchars=\\\{\}]
\PYG{k+kn}{import} \PYG{n+nn}{matplotlib}\PYG{n+nn}{.}\PYG{n+nn}{pyplot} \PYG{k}{as} \PYG{n+nn}{plt}
\PYG{k+kn}{import} \PYG{n+nn}{numpy} \PYG{k}{as} \PYG{n+nn}{np}
\PYG{o}{\PYGZpc{}}\PYG{k}{config} InlineBackend.figure\PYGZus{}format = \PYGZsq{}svg\PYGZsq{}
\PYG{n}{plt}\PYG{o}{.}\PYG{n}{rcParams}\PYG{p}{[}\PYG{l+s+s1}{\PYGZsq{}}\PYG{l+s+s1}{figure.figsize}\PYG{l+s+s1}{\PYGZsq{}}\PYG{p}{]} \PYG{o}{=} \PYG{p}{(}\PYG{l+m+mi}{13}\PYG{p}{,} \PYG{l+m+mi}{8}\PYG{p}{)}
\PYG{n}{plt}\PYG{o}{.}\PYG{n}{rcParams}\PYG{o}{.}\PYG{n}{update}\PYG{p}{(}\PYG{p}{\PYGZob{}}\PYG{l+s+s1}{\PYGZsq{}}\PYG{l+s+s1}{font.size}\PYG{l+s+s1}{\PYGZsq{}}\PYG{p}{:} \PYG{l+m+mi}{19}\PYG{p}{\PYGZcb{}}\PYG{p}{)}

\PYG{k}{def} \PYG{n+nf}{f}\PYG{p}{(}\PYG{n}{x}\PYG{p}{)}\PYG{p}{:}
    \PYG{k}{return} \PYG{n}{np}\PYG{o}{.}\PYG{n}{sin}\PYG{p}{(}\PYG{n}{x}\PYG{p}{)}

\PYG{k}{def} \PYG{n+nf}{trap\PYGZus{}plot}\PYG{p}{(}\PYG{n}{n\PYGZus{}points}\PYG{p}{)}\PYG{p}{:}
    \PYG{n}{x} \PYG{o}{=} \PYG{n}{np}\PYG{o}{.}\PYG{n}{linspace}\PYG{p}{(}\PYG{l+m+mi}{0}\PYG{p}{,} \PYG{n}{np}\PYG{o}{.}\PYG{n}{pi}\PYG{p}{,} \PYG{l+m+mi}{1000}\PYG{p}{)}          \PYG{c+c1}{\PYGZsh{} continuous}
    \PYG{n}{x\PYGZus{}i} \PYG{o}{=} \PYG{n}{np}\PYG{o}{.}\PYG{n}{linspace}\PYG{p}{(}\PYG{l+m+mi}{0}\PYG{p}{,} \PYG{n}{np}\PYG{o}{.}\PYG{n}{pi}\PYG{p}{,} \PYG{n}{n\PYGZus{}points}\PYG{p}{)}    \PYG{c+c1}{\PYGZsh{} discrete}

    \PYG{n}{plt}\PYG{o}{.}\PYG{n}{plot}\PYG{p}{(}\PYG{n}{x}\PYG{p}{,} \PYG{n}{f}\PYG{p}{(}\PYG{n}{x}\PYG{p}{)}\PYG{p}{,} \PYG{n}{label}\PYG{o}{=}\PYG{l+s+sa}{r}\PYG{l+s+s1}{\PYGZsq{}}\PYG{l+s+s1}{\PYGZdl{}}\PYG{l+s+s1}{\PYGZbs{}}\PYG{l+s+s1}{sin(x)\PYGZdl{}}\PYG{l+s+s1}{\PYGZsq{}}\PYG{p}{)}
    \PYG{n}{plt}\PYG{o}{.}\PYG{n}{plot}\PYG{p}{(}\PYG{n}{x\PYGZus{}i}\PYG{p}{,} \PYG{n}{f}\PYG{p}{(}\PYG{n}{x\PYGZus{}i}\PYG{p}{)}\PYG{p}{,} \PYG{l+s+s1}{\PYGZsq{}}\PYG{l+s+s1}{\PYGZhy{}o}\PYG{l+s+s1}{\PYGZsq{}}\PYG{p}{,} \PYG{n}{label}\PYG{o}{=}\PYG{l+s+sa}{r}\PYG{l+s+s1}{\PYGZsq{}}\PYG{l+s+s1}{\PYGZdl{}Trap\PYGZdl{} (}\PYG{l+s+si}{\PYGZob{}\PYGZcb{}}\PYG{l+s+s1}{)}\PYG{l+s+s1}{\PYGZsq{}}\PYG{o}{.}\PYG{n}{format}\PYG{p}{(}\PYG{n}{n\PYGZus{}points}\PYG{p}{)}\PYG{p}{)}
    \PYG{n}{plt}\PYG{o}{.}\PYG{n}{fill}\PYG{p}{(}\PYG{n}{x\PYGZus{}i}\PYG{p}{,} \PYG{n}{f}\PYG{p}{(}\PYG{n}{x\PYGZus{}i}\PYG{p}{)}\PYG{p}{,} \PYG{n}{color}\PYG{o}{=}\PYG{l+s+s1}{\PYGZsq{}}\PYG{l+s+s1}{C1}\PYG{l+s+s1}{\PYGZsq{}}\PYG{p}{,} \PYG{n}{alpha}\PYG{o}{=}\PYG{l+m+mf}{0.15}\PYG{p}{)}
    \PYG{n}{plt}\PYG{o}{.}\PYG{n}{vlines}\PYG{p}{(}\PYG{n}{x\PYGZus{}i}\PYG{p}{,} \PYG{l+m+mi}{0}\PYG{p}{,} \PYG{n}{f}\PYG{p}{(}\PYG{n}{x\PYGZus{}i}\PYG{p}{)}\PYG{p}{,} \PYG{n}{color}\PYG{o}{=}\PYG{l+s+s1}{\PYGZsq{}}\PYG{l+s+s1}{C1}\PYG{l+s+s1}{\PYGZsq{}}\PYG{p}{,} \PYG{n}{linestyle}\PYG{o}{=}\PYG{l+s+s1}{\PYGZsq{}}\PYG{l+s+s1}{:}\PYG{l+s+s1}{\PYGZsq{}}\PYG{p}{)}
    \PYG{n}{plt}\PYG{o}{.}\PYG{n}{xticks}\PYG{p}{(}\PYG{n}{x\PYGZus{}i}\PYG{p}{,} \PYG{p}{[}\PYG{l+s+sa}{r}\PYG{l+s+s1}{\PYGZsq{}}\PYG{l+s+s1}{\PYGZdl{}x\PYGZus{}}\PYG{l+s+si}{\PYGZob{}\PYGZcb{}}\PYG{l+s+s1}{\PYGZdl{}}\PYG{l+s+s1}{\PYGZsq{}}\PYG{o}{.}\PYG{n}{format}\PYG{p}{(}\PYG{n}{n}\PYG{p}{)} \PYG{k}{for} \PYG{n}{n} \PYG{o+ow}{in} \PYG{n+nb}{range}\PYG{p}{(}\PYG{n}{n\PYGZus{}points}\PYG{p}{)}\PYG{p}{]}\PYG{p}{)}
    \PYG{n}{plt}\PYG{o}{.}\PYG{n}{yticks}\PYG{p}{(}\PYG{p}{[}\PYG{l+m+mi}{0}\PYG{p}{,} \PYG{l+m+mi}{1}\PYG{p}{]}\PYG{p}{,} \PYG{p}{[}\PYG{l+s+s1}{\PYGZsq{}}\PYG{l+s+s1}{\PYGZdl{}0\PYGZdl{}}\PYG{l+s+s1}{\PYGZsq{}}\PYG{p}{,} \PYG{l+s+s1}{\PYGZsq{}}\PYG{l+s+s1}{\PYGZdl{}1\PYGZdl{}}\PYG{l+s+s1}{\PYGZsq{}}\PYG{p}{]}\PYG{p}{)}
    \PYG{n}{plt}\PYG{o}{.}\PYG{n}{legend}\PYG{p}{(}\PYG{n}{loc}\PYG{o}{=}\PYG{l+s+s1}{\PYGZsq{}}\PYG{l+s+s1}{best}\PYG{l+s+s1}{\PYGZsq{}}\PYG{p}{)}
    \PYG{n}{plt}\PYG{o}{.}\PYG{n}{ylim}\PYG{p}{(}\PYG{l+m+mi}{0}\PYG{p}{,} \PYG{l+m+mf}{1.05}\PYG{p}{)}
    \PYG{n}{plt}\PYG{o}{.}\PYG{n}{show}\PYG{p}{(}\PYG{p}{)}
\end{sphinxVerbatim}

\end{sphinxuseclass}\end{sphinxVerbatimInput}

\end{sphinxuseclass}
\begin{sphinxuseclass}{cell}\begin{sphinxVerbatimInput}

\begin{sphinxuseclass}{cell_input}
\begin{sphinxVerbatim}[commandchars=\\\{\}]
\PYG{n}{trap\PYGZus{}plot}\PYG{p}{(}\PYG{l+m+mi}{9}\PYG{p}{)}
\end{sphinxVerbatim}

\end{sphinxuseclass}\end{sphinxVerbatimInput}
\begin{sphinxVerbatimOutput}

\begin{sphinxuseclass}{cell_output}
\begin{sphinxVerbatim}[commandchars=\\\{\}]
\PYGZlt{}Figure size 1300x800 with 1 Axes\PYGZgt{}
\end{sphinxVerbatim}

\end{sphinxuseclass}\end{sphinxVerbatimOutput}

\end{sphinxuseclass}
\sphinxAtStartPar
Area of a Trapezoid

\sphinxAtStartPar
The area of a trapezoid is the average height times the base



\sphinxAtStartPar
For the example shown, the area is
\begin{equation*}
\begin{split}
Area = \overbrace{\frac{f(x_{k-1}) + f(x_k)}{2}}^{ave \, height} \,\, \overbrace{\Delta x}^{base}
\end{split}
\end{equation*}
\sphinxAtStartPar
trapezoid rule equation

\sphinxAtStartPar
the equation for the trapezoid rule is
\begin{equation*}
\begin{split}
\int_a^b f(x)dx \approx \frac{f(x_0) + f(x_1)}{2}\Delta x + \frac{f(x_1) + f(x_2)}{2}\Delta x \cdots \frac{f(x_{N-1}) + f(x_N)}{2}\Delta x
\end{split}
\end{equation*}
\sphinxAtStartPar
Now we can write this as a summation
\$\(
\approx \sum_{k=1}^N \frac{f(x_{k-1}) + f(x_k)}{2} \Delta x
\)\$

\sphinxAtStartPar
And we can write a function to compute this
\begin{equation*}
\begin{split}
\approx \sum_{k=1}^N \frac{f(x_{k-1}) + f(x_k)}{2} \Delta x
\end{split}
\end{equation*}
\begin{sphinxuseclass}{cell}\begin{sphinxVerbatimInput}

\begin{sphinxuseclass}{cell_input}
\begin{sphinxVerbatim}[commandchars=\\\{\}]
\PYG{k}{def} \PYG{n+nf}{trap}\PYG{p}{(}\PYG{n}{f}\PYG{p}{,} \PYG{n}{x}\PYG{p}{)}\PYG{p}{:}
\PYG{+w}{    }\PYG{l+s+sd}{\PYGZdq{}\PYGZdq{}\PYGZdq{}}
\PYG{l+s+sd}{    computes the integral of f using trapezoid rule}
\PYG{l+s+sd}{    \PYGZdq{}\PYGZdq{}\PYGZdq{}}
    \PYG{n}{area} \PYG{o}{=} \PYG{l+m+mi}{0}
    \PYG{n}{N} \PYG{o}{=} \PYG{n+nb}{len}\PYG{p}{(}\PYG{n}{x}\PYG{p}{)}
    \PYG{n}{dx} \PYG{o}{=} \PYG{n}{x}\PYG{p}{[}\PYG{l+m+mi}{1}\PYG{p}{]} \PYG{o}{\PYGZhy{}} \PYG{n}{x}\PYG{p}{[}\PYG{l+m+mi}{0}\PYG{p}{]}
    
    \PYG{k}{for} \PYG{n}{k} \PYG{o+ow}{in} \PYG{n+nb}{range}\PYG{p}{(}\PYG{l+m+mi}{1}\PYG{p}{,} \PYG{n}{N}\PYG{p}{)}\PYG{p}{:}
        \PYG{n}{area} \PYG{o}{+}\PYG{o}{=} \PYG{p}{(}\PYG{n}{f}\PYG{p}{(}\PYG{n}{x}\PYG{p}{[}\PYG{n}{k} \PYG{o}{\PYGZhy{}} \PYG{l+m+mi}{1}\PYG{p}{]}\PYG{p}{)} \PYG{o}{+} \PYG{n}{f}\PYG{p}{(}\PYG{n}{x}\PYG{p}{[}\PYG{n}{k}\PYG{p}{]}\PYG{p}{)}\PYG{p}{)} \PYG{o}{*} \PYG{n}{dx} \PYG{o}{/} \PYG{l+m+mi}{2}
        
    \PYG{k}{return} \PYG{n}{area}
\end{sphinxVerbatim}

\end{sphinxuseclass}\end{sphinxVerbatimInput}

\end{sphinxuseclass}
\begin{sphinxuseclass}{cell}\begin{sphinxVerbatimInput}

\begin{sphinxuseclass}{cell_input}
\begin{sphinxVerbatim}[commandchars=\\\{\}]
\PYG{n}{x} \PYG{o}{=} \PYG{n}{np}\PYG{o}{.}\PYG{n}{linspace}\PYG{p}{(}\PYG{l+m+mi}{0}\PYG{p}{,} \PYG{n}{np}\PYG{o}{.}\PYG{n}{pi}\PYG{p}{,} \PYG{l+m+mi}{20}\PYG{p}{)}
\PYG{n}{trap}\PYG{p}{(}\PYG{n}{f}\PYG{p}{,} \PYG{n}{x}\PYG{p}{)}
\end{sphinxVerbatim}

\end{sphinxuseclass}\end{sphinxVerbatimInput}
\begin{sphinxVerbatimOutput}

\begin{sphinxuseclass}{cell_output}
\begin{sphinxVerbatim}[commandchars=\\\{\}]
1.9954413183201944
\end{sphinxVerbatim}

\end{sphinxuseclass}\end{sphinxVerbatimOutput}

\end{sphinxuseclass}
\sphinxAtStartPar
Error vs. number of Trapezoids

\sphinxAtStartPar
As we increase the number of trapezoids, the approximation gets better (\(error \rightarrow 0\)).
\$\(error = (actual - approximation)^2\)\$

\begin{sphinxuseclass}{cell}\begin{sphinxVerbatimInput}

\begin{sphinxuseclass}{cell_input}
\begin{sphinxVerbatim}[commandchars=\\\{\}]
\PYG{n}{plt}\PYG{o}{.}\PYG{n}{rcParams}\PYG{p}{[}\PYG{l+s+s1}{\PYGZsq{}}\PYG{l+s+s1}{figure.figsize}\PYG{l+s+s1}{\PYGZsq{}}\PYG{p}{]} \PYG{o}{=} \PYG{p}{(}\PYG{l+m+mi}{11}\PYG{p}{,} \PYG{l+m+mi}{6}\PYG{p}{)}
\PYG{k}{def} \PYG{n+nf}{plot\PYGZus{}error}\PYG{p}{(}\PYG{n}{n\PYGZus{}points}\PYG{p}{)}\PYG{p}{:}
    \PYG{k}{for} \PYG{n}{n} \PYG{o+ow}{in} \PYG{n+nb}{range}\PYG{p}{(}\PYG{l+m+mi}{2}\PYG{p}{,} \PYG{n}{n\PYGZus{}points}\PYG{p}{)}\PYG{p}{:}
        \PYG{n}{x} \PYG{o}{=} \PYG{n}{np}\PYG{o}{.}\PYG{n}{linspace}\PYG{p}{(}\PYG{l+m+mi}{0}\PYG{p}{,} \PYG{n}{np}\PYG{o}{.}\PYG{n}{pi}\PYG{p}{,} \PYG{n}{n}\PYG{p}{)}
        \PYG{n}{plt}\PYG{o}{.}\PYG{n}{plot}\PYG{p}{(}\PYG{n}{n} \PYG{o}{\PYGZhy{}} \PYG{l+m+mi}{1}\PYG{p}{,} \PYG{p}{(}\PYG{n}{trap}\PYG{p}{(}\PYG{n}{f}\PYG{p}{,} \PYG{n}{x}\PYG{p}{)} \PYG{o}{\PYGZhy{}} \PYG{l+m+mi}{2}\PYG{p}{)} \PYG{o}{*}\PYG{o}{*} \PYG{l+m+mi}{2}\PYG{p}{,} \PYG{l+s+s1}{\PYGZsq{}}\PYG{l+s+s1}{bo}\PYG{l+s+s1}{\PYGZsq{}}\PYG{p}{)}

        \PYG{n}{plt}\PYG{o}{.}\PYG{n}{axhline}\PYG{p}{(}\PYG{l+m+mi}{0}\PYG{p}{,} \PYG{n}{color}\PYG{o}{=}\PYG{l+s+s1}{\PYGZsq{}}\PYG{l+s+s1}{black}\PYG{l+s+s1}{\PYGZsq{}}\PYG{p}{,} \PYG{n}{lw}\PYG{o}{=}\PYG{l+m+mi}{1}\PYG{p}{)}
        \PYG{n}{plt}\PYG{o}{.}\PYG{n}{xlabel}\PYG{p}{(}\PYG{l+s+s1}{\PYGZsq{}}\PYG{l+s+s1}{\PYGZsh{} of trapezoids}\PYG{l+s+s1}{\PYGZsq{}}\PYG{p}{)}
        \PYG{n}{plt}\PYG{o}{.}\PYG{n}{ylabel}\PYG{p}{(}\PYG{l+s+s1}{\PYGZsq{}}\PYG{l+s+s1}{error}\PYG{l+s+s1}{\PYGZsq{}}\PYG{p}{)}
    \PYG{n}{plt}\PYG{o}{.}\PYG{n}{show}\PYG{p}{(}\PYG{p}{)}
\end{sphinxVerbatim}

\end{sphinxuseclass}\end{sphinxVerbatimInput}

\end{sphinxuseclass}
\begin{sphinxuseclass}{cell}\begin{sphinxVerbatimInput}

\begin{sphinxuseclass}{cell_input}
\begin{sphinxVerbatim}[commandchars=\\\{\}]
\PYG{n}{plot\PYGZus{}error}\PYG{p}{(}\PYG{l+m+mi}{10}\PYG{p}{)}
\end{sphinxVerbatim}

\end{sphinxuseclass}\end{sphinxVerbatimInput}
\begin{sphinxVerbatimOutput}

\begin{sphinxuseclass}{cell_output}
\begin{sphinxVerbatim}[commandchars=\\\{\}]
\PYGZlt{}Figure size 1100x600 with 1 Axes\PYGZgt{}
\end{sphinxVerbatim}

\end{sphinxuseclass}\end{sphinxVerbatimOutput}

\end{sphinxuseclass}
\sphinxstepscope


\chapter{testAnimated\sphinxhyphen{}Sinc\sphinxhyphen{}and\sphinxhyphen{}FT\sphinxhyphen{}example}
\label{\detokenize{testAnimated-Sinc-and-FT-example:testanimated-sinc-and-ft-example}}\label{\detokenize{testAnimated-Sinc-and-FT-example::doc}}
\begin{sphinxuseclass}{cell}\begin{sphinxVerbatimInput}

\begin{sphinxuseclass}{cell_input}
\begin{sphinxVerbatim}[commandchars=\\\{\}]
\PYG{k+kn}{import} \PYG{n+nn}{numpy} \PYG{k}{as} \PYG{n+nn}{np}

\PYG{k}{def} \PYG{n+nf}{rect}\PYG{p}{(}\PYG{n}{x}\PYG{p}{,} \PYG{n}{B}\PYG{p}{)}\PYG{p}{:}
\PYG{+w}{    }\PYG{l+s+sd}{\PYGZdq{}\PYGZdq{}\PYGZdq{}}
\PYG{l+s+sd}{    create a rectangle function}
\PYG{l+s+sd}{    returns a numpy array that is 1 if |x| \PYGZlt{} w and 0 if |x| \PYGZgt{} w}
\PYG{l+s+sd}{    B is the rectangle width centered at 0}
\PYG{l+s+sd}{    x is the number of points in the array}
\PYG{l+s+sd}{    \PYGZdq{}\PYGZdq{}\PYGZdq{}}
    
    \PYG{n}{B} \PYG{o}{=} \PYG{n+nb}{int}\PYG{p}{(}\PYG{n}{B}\PYG{p}{)}
    \PYG{n}{x} \PYG{o}{=} \PYG{n+nb}{int}\PYG{p}{(}\PYG{n}{x}\PYG{p}{)}
    
    \PYG{n}{high} \PYG{o}{=} \PYG{n}{np}\PYG{o}{.}\PYG{n}{ones}\PYG{p}{(}\PYG{n}{B}\PYG{p}{)}
    \PYG{n}{low1} \PYG{o}{=} \PYG{n}{np}\PYG{o}{.}\PYG{n}{zeros}\PYG{p}{(}\PYG{n+nb}{int}\PYG{p}{(}\PYG{n}{x}\PYG{o}{/}\PYG{l+m+mi}{2} \PYG{o}{\PYGZhy{}} \PYG{n}{B}\PYG{o}{/}\PYG{l+m+mi}{2}\PYG{p}{)}\PYG{p}{)}    
    \PYG{n}{x1} \PYG{o}{=} \PYG{n}{np}\PYG{o}{.}\PYG{n}{append}\PYG{p}{(}\PYG{n}{low1}\PYG{p}{,} \PYG{n}{high}\PYG{p}{)}
    \PYG{n}{rect} \PYG{o}{=} \PYG{n}{np}\PYG{o}{.}\PYG{n}{append}\PYG{p}{(}\PYG{n}{x1}\PYG{p}{,} \PYG{n}{low1}\PYG{p}{)}
    
    \PYG{k}{if} \PYG{n}{x} \PYG{o}{\PYGZgt{}} \PYG{n+nb}{len}\PYG{p}{(}\PYG{n}{rect}\PYG{p}{)}\PYG{p}{:}
        \PYG{n}{rect} \PYG{o}{=} \PYG{n}{np}\PYG{o}{.}\PYG{n}{append}\PYG{p}{(}\PYG{n}{rect}\PYG{p}{,} \PYG{l+m+mi}{0}\PYG{p}{)}
    \PYG{k}{elif} \PYG{n}{x} \PYG{o}{\PYGZlt{}} \PYG{n+nb}{len}\PYG{p}{(}\PYG{n}{rect}\PYG{p}{)}\PYG{p}{:}
        \PYG{n}{rect} \PYG{o}{=} \PYG{n}{rect}\PYG{p}{[}\PYG{p}{:}\PYG{o}{\PYGZhy{}}\PYG{l+m+mi}{1}\PYG{p}{]}

    \PYG{k}{return} \PYG{n}{rect}
\end{sphinxVerbatim}

\end{sphinxuseclass}\end{sphinxVerbatimInput}

\end{sphinxuseclass}
\begin{sphinxuseclass}{cell}\begin{sphinxVerbatimInput}

\begin{sphinxuseclass}{cell_input}
\begin{sphinxVerbatim}[commandchars=\\\{\}]
\PYG{o}{\PYGZpc{}}\PYG{k}{matplotlib} notebook
\PYG{k+kn}{import} \PYG{n+nn}{matplotlib}\PYG{n+nn}{.}\PYG{n+nn}{pyplot} \PYG{k}{as} \PYG{n+nn}{plt}
\PYG{k+kn}{from} \PYG{n+nn}{matplotlib}\PYG{n+nn}{.}\PYG{n+nn}{animation} \PYG{k+kn}{import} \PYG{n}{FuncAnimation}

\PYG{c+c1}{\PYGZsh{} constants and x array}
\PYG{n}{pi} \PYG{o}{=} \PYG{n}{np}\PYG{o}{.}\PYG{n}{pi}
\PYG{n}{length} \PYG{o}{=} \PYG{l+m+mi}{2000}
\PYG{n}{x} \PYG{o}{=} \PYG{n}{np}\PYG{o}{.}\PYG{n}{linspace}\PYG{p}{(}\PYG{o}{\PYGZhy{}}\PYG{l+m+mi}{1}\PYG{p}{,} \PYG{l+m+mi}{1}\PYG{p}{,} \PYG{n}{length}\PYG{p}{)}

\PYG{c+c1}{\PYGZsh{} create figure and axes }
\PYG{n}{fig}\PYG{p}{,} \PYG{p}{(}\PYG{n}{ax1}\PYG{p}{,} \PYG{n}{ax2}\PYG{p}{)} \PYG{o}{=} \PYG{n}{plt}\PYG{o}{.}\PYG{n}{subplots}\PYG{p}{(}\PYG{l+m+mi}{2}\PYG{p}{,} \PYG{n}{figsize}\PYG{o}{=}\PYG{p}{(}\PYG{l+m+mi}{12}\PYG{p}{,} \PYG{l+m+mi}{6}\PYG{p}{)}\PYG{p}{)}

\PYG{c+c1}{\PYGZsh{} creating our line objects for the plots}
\PYG{n}{sinc}\PYG{p}{,} \PYG{o}{=} \PYG{n}{ax1}\PYG{o}{.}\PYG{n}{plot}\PYG{p}{(}\PYG{n}{x}\PYG{p}{,} \PYG{n}{np}\PYG{o}{.}\PYG{n}{sin}\PYG{p}{(}\PYG{n}{x}\PYG{p}{)}\PYG{p}{,} \PYG{l+s+s1}{\PYGZsq{}}\PYG{l+s+s1}{\PYGZhy{}b}\PYG{l+s+s1}{\PYGZsq{}}\PYG{p}{)}
\PYG{n}{box}\PYG{p}{,} \PYG{o}{=} \PYG{n}{ax2}\PYG{o}{.}\PYG{n}{plot}\PYG{p}{(}\PYG{n}{x}\PYG{p}{,} \PYG{n}{np}\PYG{o}{.}\PYG{n}{sin}\PYG{p}{(}\PYG{n}{x}\PYG{p}{)}\PYG{p}{,} \PYG{l+s+s1}{\PYGZsq{}}\PYG{l+s+s1}{\PYGZhy{}r}\PYG{l+s+s1}{\PYGZsq{}}\PYG{p}{)}

\PYG{k}{def} \PYG{n+nf}{animate}\PYG{p}{(}\PYG{n}{B}\PYG{p}{)}\PYG{p}{:}
\PYG{+w}{    }\PYG{l+s+sd}{\PYGZdq{}\PYGZdq{}\PYGZdq{}}
\PYG{l+s+sd}{    this function gets called by FuncAnimation}
\PYG{l+s+sd}{    each time called, it will replot with a different width \PYGZdq{}B\PYGZdq{}}
\PYG{l+s+sd}{    }
\PYG{l+s+sd}{    B: rect width}
\PYG{l+s+sd}{    }
\PYG{l+s+sd}{    return:}
\PYG{l+s+sd}{        sinc: ydata}
\PYG{l+s+sd}{        box: ydata}
\PYG{l+s+sd}{    \PYGZdq{}\PYGZdq{}\PYGZdq{}}
    
    \PYG{c+c1}{\PYGZsh{} create our rect object}
    \PYG{n}{f} \PYG{o}{=} \PYG{n}{rect}\PYG{p}{(}\PYG{n+nb}{len}\PYG{p}{(}\PYG{n}{x}\PYG{p}{)}\PYG{p}{,} \PYG{n}{B}\PYG{p}{)}
    \PYG{n}{box}\PYG{o}{.}\PYG{n}{set\PYGZus{}ydata}\PYG{p}{(}\PYG{n}{f}\PYG{p}{)}
    
    \PYG{c+c1}{\PYGZsh{} create our sinc object}
    \PYG{n}{F} \PYG{o}{=} \PYG{p}{(}\PYG{n}{B} \PYG{o}{/} \PYG{n}{length}\PYG{p}{)} \PYG{o}{*} \PYG{n}{np}\PYG{o}{.}\PYG{n}{sin}\PYG{p}{(}\PYG{n}{x} \PYG{o}{*} \PYG{n}{B} \PYG{o}{/} \PYG{l+m+mi}{2}\PYG{p}{)} \PYG{o}{/} \PYG{p}{(}\PYG{n}{x} \PYG{o}{*} \PYG{n}{B} \PYG{o}{/} \PYG{l+m+mi}{2}\PYG{p}{)}
    \PYG{n}{sinc}\PYG{o}{.}\PYG{n}{set\PYGZus{}ydata}\PYG{p}{(}\PYG{n}{F}\PYG{p}{)}
    
    \PYG{c+c1}{\PYGZsh{} adjust the sinc plot height in a loop}
    \PYG{n}{ax1}\PYG{o}{.}\PYG{n}{set\PYGZus{}ylim}\PYG{p}{(}\PYG{n}{np}\PYG{o}{.}\PYG{n}{min}\PYG{p}{(}\PYG{n}{F}\PYG{p}{)}\PYG{p}{,} \PYG{n}{np}\PYG{o}{.}\PYG{n}{max}\PYG{p}{(}\PYG{n}{F}\PYG{p}{)}\PYG{p}{)}
    
    \PYG{c+c1}{\PYGZsh{} format the ax1 yticks}
    \PYG{n}{plt}\PYG{o}{.}\PYG{n}{setp}\PYG{p}{(}\PYG{n}{ax1}\PYG{p}{,} \PYG{n}{xticks}\PYG{o}{=}\PYG{p}{[}\PYG{o}{\PYGZhy{}}\PYG{l+m+mf}{0.25}\PYG{p}{,} \PYG{l+m+mf}{0.25}\PYG{p}{]}\PYG{p}{,} \PYG{n}{xticklabels}\PYG{o}{=}\PYG{p}{[}\PYG{l+s+s1}{\PYGZsq{}}\PYG{l+s+s1}{\PYGZhy{}1/4}\PYG{l+s+s1}{\PYGZsq{}}\PYG{p}{,} \PYG{l+s+s1}{\PYGZsq{}}\PYG{l+s+s1}{1/4}\PYG{l+s+s1}{\PYGZsq{}}\PYG{p}{]}\PYG{p}{,}
             \PYG{n}{yticks}\PYG{o}{=}\PYG{p}{[}\PYG{l+m+mi}{0}\PYG{p}{,} \PYG{n}{np}\PYG{o}{.}\PYG{n}{max}\PYG{p}{(}\PYG{n}{F}\PYG{p}{)}\PYG{p}{]}\PYG{p}{,} \PYG{n}{yticklabels}\PYG{o}{=}\PYG{p}{[}\PYG{l+s+s1}{\PYGZsq{}}\PYG{l+s+s1}{0}\PYG{l+s+s1}{\PYGZsq{}}\PYG{p}{,} \PYG{l+s+s1}{\PYGZsq{}}\PYG{l+s+s1}{B=}\PYG{l+s+si}{\PYGZob{}:.2f\PYGZcb{}}\PYG{l+s+s1}{\PYGZsq{}}\PYG{o}{.}\PYG{n}{format}\PYG{p}{(}\PYG{p}{(}\PYG{n}{B} \PYG{o}{/} \PYG{n}{length}\PYG{p}{)}\PYG{p}{)}\PYG{p}{]}\PYG{p}{)}
    
    \PYG{c+c1}{\PYGZsh{} format the ax2 xticks to move with the box}
    \PYG{n}{plt}\PYG{o}{.}\PYG{n}{setp}\PYG{p}{(}\PYG{n}{ax2}\PYG{p}{,} \PYG{n}{yticks}\PYG{o}{=}\PYG{p}{[}\PYG{l+m+mi}{0}\PYG{p}{,} \PYG{l+m+mi}{1}\PYG{p}{]}\PYG{p}{,} 
             \PYG{n}{xticks}\PYG{o}{=}\PYG{p}{[}\PYG{o}{\PYGZhy{}}\PYG{l+m+mi}{1}\PYG{p}{,} \PYG{o}{\PYGZhy{}}\PYG{l+m+mi}{1} \PYG{o}{*} \PYG{n}{B} \PYG{o}{/} \PYG{n}{length}\PYG{p}{,} \PYG{l+m+mi}{1} \PYG{o}{*} \PYG{n}{B} \PYG{o}{/} \PYG{n}{length}\PYG{p}{,} \PYG{l+m+mi}{1}\PYG{p}{]}\PYG{p}{,} \PYG{n}{xticklabels}\PYG{o}{=}\PYG{p}{[}\PYG{l+s+s1}{\PYGZsq{}}\PYG{l+s+s1}{\PYGZhy{}1}\PYG{l+s+s1}{\PYGZsq{}}\PYG{p}{,} \PYG{l+s+s1}{\PYGZsq{}}\PYG{l+s+s1}{\PYGZhy{}B/2}\PYG{l+s+s1}{\PYGZsq{}}\PYG{p}{,} \PYG{l+s+s1}{\PYGZsq{}}\PYG{l+s+s1}{B/2}\PYG{l+s+s1}{\PYGZsq{}}\PYG{p}{,} \PYG{l+s+s1}{\PYGZsq{}}\PYG{l+s+s1}{1}\PYG{l+s+s1}{\PYGZsq{}}\PYG{p}{]}\PYG{p}{)}
    
\PYG{k}{def} \PYG{n+nf}{init}\PYG{p}{(}\PYG{p}{)}\PYG{p}{:}
\PYG{+w}{    }\PYG{l+s+sd}{\PYGZdq{}\PYGZdq{}\PYGZdq{}}
\PYG{l+s+sd}{    initialize the figure}
\PYG{l+s+sd}{    \PYGZdq{}\PYGZdq{}\PYGZdq{}}
    
    \PYG{n}{ax2}\PYG{o}{.}\PYG{n}{set\PYGZus{}ylim}\PYG{p}{(}\PYG{o}{\PYGZhy{}}\PYG{l+m+mf}{0.2}\PYG{p}{,} \PYG{l+m+mf}{1.1}\PYG{p}{)}
    \PYG{n}{ax1}\PYG{o}{.}\PYG{n}{set\PYGZus{}xlim}\PYG{p}{(}\PYG{o}{\PYGZhy{}}\PYG{l+m+mf}{0.25}\PYG{p}{,} \PYG{l+m+mf}{0.25}\PYG{p}{)}
    \PYG{n}{ax2}\PYG{o}{.}\PYG{n}{set\PYGZus{}xlim}\PYG{p}{(}\PYG{o}{\PYGZhy{}}\PYG{l+m+mi}{1}\PYG{p}{,} \PYG{l+m+mi}{1}\PYG{p}{)}
    \PYG{n}{ax1}\PYG{o}{.}\PYG{n}{axhline}\PYG{p}{(}\PYG{l+m+mi}{0}\PYG{p}{,} \PYG{n}{color}\PYG{o}{=}\PYG{l+s+s1}{\PYGZsq{}}\PYG{l+s+s1}{black}\PYG{l+s+s1}{\PYGZsq{}}\PYG{p}{,} \PYG{n}{lw}\PYG{o}{=}\PYG{l+m+mi}{1}\PYG{p}{)}
    \PYG{n}{ax2}\PYG{o}{.}\PYG{n}{axhline}\PYG{p}{(}\PYG{l+m+mi}{0}\PYG{p}{,} \PYG{n}{color}\PYG{o}{=}\PYG{l+s+s1}{\PYGZsq{}}\PYG{l+s+s1}{black}\PYG{l+s+s1}{\PYGZsq{}}\PYG{p}{,} \PYG{n}{lw}\PYG{o}{=}\PYG{l+m+mi}{1}\PYG{p}{)}
    \PYG{n}{plt}\PYG{o}{.}\PYG{n}{rcParams}\PYG{o}{.}\PYG{n}{update}\PYG{p}{(}\PYG{p}{\PYGZob{}}\PYG{l+s+s1}{\PYGZsq{}}\PYG{l+s+s1}{font.size}\PYG{l+s+s1}{\PYGZsq{}}\PYG{p}{:}\PYG{l+m+mi}{14}\PYG{p}{\PYGZcb{}}\PYG{p}{)}
    
    \PYG{k}{return} \PYG{n}{sinc}\PYG{p}{,} \PYG{n}{box}\PYG{p}{,}

\PYG{c+c1}{\PYGZsh{} the FuncAnimation function iterates through our animate function using the steps array}
\PYG{n}{step} \PYG{o}{=} \PYG{l+m+mi}{10}
\PYG{n}{steps} \PYG{o}{=} \PYG{n}{np}\PYG{o}{.}\PYG{n}{append}\PYG{p}{(}\PYG{n}{np}\PYG{o}{.}\PYG{n}{arange}\PYG{p}{(}\PYG{l+m+mi}{10}\PYG{p}{,} \PYG{l+m+mi}{1000}\PYG{p}{,} \PYG{n}{step}\PYG{p}{)}\PYG{p}{,} \PYG{n}{np}\PYG{o}{.}\PYG{n}{arange}\PYG{p}{(}\PYG{l+m+mi}{1000}\PYG{p}{,} \PYG{l+m+mi}{10}\PYG{p}{,} \PYG{o}{\PYGZhy{}}\PYG{l+m+mi}{1} \PYG{o}{*} \PYG{n}{step}\PYG{p}{)}\PYG{p}{)}
\PYG{n}{ani} \PYG{o}{=} \PYG{n}{FuncAnimation}\PYG{p}{(}\PYG{n}{fig}\PYG{p}{,} \PYG{n}{animate}\PYG{p}{,} \PYG{n}{steps}\PYG{p}{,} \PYG{n}{init\PYGZus{}func}\PYG{o}{=}\PYG{n}{init}\PYG{p}{,} \PYG{n}{interval}\PYG{o}{=}\PYG{l+m+mi}{50}\PYG{p}{,} \PYG{n}{blit}\PYG{o}{=}\PYG{k+kc}{True}\PYG{p}{)}
\PYG{n}{plt}\PYG{o}{.}\PYG{n}{show}\PYG{p}{(}\PYG{p}{)}
\end{sphinxVerbatim}

\end{sphinxuseclass}\end{sphinxVerbatimInput}
\begin{sphinxVerbatimOutput}

\begin{sphinxuseclass}{cell_output}
\begin{sphinxVerbatim}[commandchars=\\\{\}]
\PYGZlt{}IPython.core.display.Javascript object\PYGZgt{}
\end{sphinxVerbatim}

\begin{sphinxVerbatim}[commandchars=\\\{\}]
\PYGZlt{}IPython.core.display.HTML object\PYGZgt{}
\end{sphinxVerbatim}

\end{sphinxuseclass}\end{sphinxVerbatimOutput}

\end{sphinxuseclass}
\sphinxstepscope


\chapter{testNewton\sphinxhyphen{}outdatewarning}
\label{\detokenize{testNewton-outdatewarning:testnewton-outdatewarning}}\label{\detokenize{testNewton-outdatewarning::doc}}

\chapter{Newton’s Method}
\label{\detokenize{testNewton-outdatewarning:newton-s-method}}
\sphinxAtStartPar
for finding roots


\chapter{What is Newton’s Method}
\label{\detokenize{testNewton-outdatewarning:what-is-newton-s-method}}
\sphinxAtStartPar
Newton’s method is an iterative process for finding the roots of a function.


\chapter{Concept}
\label{\detokenize{testNewton-outdatewarning:concept}}
\sphinxAtStartPar
The concept is to
\begin{itemize}
\item {} 
\sphinxAtStartPar
guess a starting \(x\) point \((x_1)\)

\item {} 
\sphinxAtStartPar
find a linear equation that’s tangent to and passes through \(f(x_1)\)

\item {} 
\sphinxAtStartPar
move to the x intercept

\item {} 
\sphinxAtStartPar
repeat

\end{itemize}

\begin{sphinxuseclass}{cell}\begin{sphinxVerbatimInput}

\begin{sphinxuseclass}{cell_input}
\begin{sphinxVerbatim}[commandchars=\\\{\}]
\PYG{k+kn}{import} \PYG{n+nn}{numpy} \PYG{k}{as} \PYG{n+nn}{np}
\PYG{k+kn}{from} \PYG{n+nn}{scipy}\PYG{n+nn}{.}\PYG{n+nn}{misc} \PYG{k+kn}{import} \PYG{n}{derivative}
\PYG{k+kn}{import} \PYG{n+nn}{matplotlib}\PYG{n+nn}{.}\PYG{n+nn}{pyplot} \PYG{k}{as} \PYG{n+nn}{plt}
\PYG{k+kn}{from} \PYG{n+nn}{ipywidgets} \PYG{k+kn}{import} \PYG{n}{widgets}
\PYG{o}{\PYGZpc{}}\PYG{k}{matplotlib} nbagg

\PYG{n}{x} \PYG{o}{=} \PYG{n}{np}\PYG{o}{.}\PYG{n}{linspace}\PYG{p}{(}\PYG{l+m+mf}{0.2}\PYG{p}{,} \PYG{l+m+mf}{2.2}\PYG{p}{,} \PYG{l+m+mi}{500}\PYG{p}{)}

\PYG{k}{def} \PYG{n+nf}{f}\PYG{p}{(}\PYG{n}{x}\PYG{p}{)}\PYG{p}{:}
    \PYG{k}{return} \PYG{n}{np}\PYG{o}{.}\PYG{n}{log}\PYG{p}{(}\PYG{n}{x}\PYG{p}{)}

\PYG{k}{def} \PYG{n+nf}{f\PYGZus{}line}\PYG{p}{(}\PYG{n}{f}\PYG{p}{,} \PYG{n}{x}\PYG{p}{,} \PYG{n}{x\PYGZus{}n}\PYG{p}{)}\PYG{p}{:}
    \PYG{n}{slope} \PYG{o}{=} \PYG{n}{derivative}\PYG{p}{(}\PYG{n}{f}\PYG{p}{,} \PYG{n}{x\PYGZus{}n}\PYG{p}{,} \PYG{n}{dx}\PYG{o}{=}\PYG{l+m+mf}{0.1}\PYG{p}{)}
    \PYG{n}{x\PYGZus{}nn} \PYG{o}{=} \PYG{n}{x\PYGZus{}n} \PYG{o}{\PYGZhy{}} \PYG{n}{f}\PYG{p}{(}\PYG{n}{x\PYGZus{}n}\PYG{p}{)} \PYG{o}{/} \PYG{n}{slope} 
    \PYG{k}{return} \PYG{n}{slope} \PYG{o}{*} \PYG{p}{(}\PYG{n}{x} \PYG{o}{\PYGZhy{}} \PYG{n}{x\PYGZus{}n}\PYG{p}{)} \PYG{o}{+} \PYG{n}{f}\PYG{p}{(}\PYG{n}{x\PYGZus{}n}\PYG{p}{)}\PYG{p}{,} \PYG{n}{x\PYGZus{}nn}

\PYG{k}{def} \PYG{n+nf}{update\PYGZus{}plot}\PYG{p}{(}\PYG{n}{order}\PYG{p}{)}\PYG{p}{:}
    \PYG{n}{x\PYGZus{}n} \PYG{o}{=} \PYG{l+m+mi}{2}
    \PYG{n}{ax}\PYG{o}{.}\PYG{n}{clear}\PYG{p}{(}\PYG{p}{)}
    \PYG{n}{ax}\PYG{o}{.}\PYG{n}{plot}\PYG{p}{(}\PYG{n}{x}\PYG{p}{,} \PYG{n}{f}\PYG{p}{(}\PYG{n}{x}\PYG{p}{)}\PYG{p}{,} \PYG{n}{label}\PYG{o}{=}\PYG{l+s+sa}{r}\PYG{l+s+s1}{\PYGZsq{}}\PYG{l+s+s1}{\PYGZdl{}}\PYG{l+s+s1}{\PYGZbs{}}\PYG{l+s+s1}{ln(x)\PYGZdl{}}\PYG{l+s+s1}{\PYGZsq{}}\PYG{p}{)}
    \PYG{n}{ax}\PYG{o}{.}\PYG{n}{axhline}\PYG{p}{(}\PYG{l+m+mi}{0}\PYG{p}{,} \PYG{n}{color}\PYG{o}{=}\PYG{l+s+s1}{\PYGZsq{}}\PYG{l+s+s1}{gray}\PYG{l+s+s1}{\PYGZsq{}}\PYG{p}{,} \PYG{n}{lw}\PYG{o}{=}\PYG{l+m+mf}{0.5}\PYG{p}{)}

    \PYG{k}{for} \PYG{n}{i} \PYG{o+ow}{in} \PYG{n+nb}{range}\PYG{p}{(}\PYG{l+m+mi}{0}\PYG{p}{,} \PYG{n}{order}\PYG{p}{)}\PYG{p}{:}

        \PYG{k}{if} \PYG{n}{i} \PYG{o}{\PYGZgt{}}\PYG{o}{=} \PYG{l+m+mi}{1}\PYG{p}{:}
            \PYG{n}{ax}\PYG{o}{.}\PYG{n}{plot}\PYG{p}{(}\PYG{n}{x}\PYG{p}{,} \PYG{n}{f\PYGZus{}l}\PYG{p}{,} \PYG{l+s+s1}{\PYGZsq{}}\PYG{l+s+s1}{\PYGZhy{}\PYGZhy{}}\PYG{l+s+s1}{\PYGZsq{}}\PYG{p}{,} \PYG{n}{lw}\PYG{o}{=}\PYG{l+m+mi}{1}\PYG{p}{)}
        
        \PYG{n}{ax}\PYG{o}{.}\PYG{n}{plot}\PYG{p}{(}\PYG{n}{x\PYGZus{}n}\PYG{p}{,} \PYG{n}{f}\PYG{p}{(}\PYG{n}{x\PYGZus{}n}\PYG{p}{)}\PYG{p}{,} \PYG{l+s+s1}{\PYGZsq{}}\PYG{l+s+s1}{kd}\PYG{l+s+s1}{\PYGZsq{}}\PYG{p}{,} \PYG{n}{label}\PYG{o}{=}\PYG{l+s+sa}{r}\PYG{l+s+s1}{\PYGZsq{}}\PYG{l+s+s1}{\PYGZdl{}f(x\PYGZus{}}\PYG{l+s+si}{\PYGZob{}\PYGZcb{}}\PYG{l+s+s1}{=}\PYG{l+s+si}{\PYGZob{}:.2f\PYGZcb{}}\PYG{l+s+s1}{)=}\PYG{l+s+si}{\PYGZob{}:.3f\PYGZcb{}}\PYG{l+s+s1}{\PYGZdl{}}\PYG{l+s+s1}{\PYGZsq{}}\PYG{o}{.}\PYG{n}{format}\PYG{p}{(}\PYG{n}{i} \PYG{o}{+} \PYG{l+m+mi}{1}\PYG{p}{,} \PYG{n}{x\PYGZus{}n}\PYG{p}{,} \PYG{n}{f}\PYG{p}{(}\PYG{n}{x\PYGZus{}n}\PYG{p}{)}\PYG{p}{)}\PYG{p}{)}
        \PYG{n}{ax}\PYG{o}{.}\PYG{n}{vlines}\PYG{p}{(}\PYG{n}{x\PYGZus{}n}\PYG{p}{,} \PYG{l+m+mi}{0}\PYG{p}{,} \PYG{n}{f}\PYG{p}{(}\PYG{n}{x\PYGZus{}n}\PYG{p}{)}\PYG{p}{,} \PYG{n}{color}\PYG{o}{=}\PYG{l+s+s1}{\PYGZsq{}}\PYG{l+s+s1}{black}\PYG{l+s+s1}{\PYGZsq{}}\PYG{p}{,} \PYG{n}{linestyle}\PYG{o}{=}\PYG{l+s+s1}{\PYGZsq{}}\PYG{l+s+s1}{:}\PYG{l+s+s1}{\PYGZsq{}}\PYG{p}{,} \PYG{n}{lw}\PYG{o}{=}\PYG{l+m+mi}{1}\PYG{p}{)}
        \PYG{n}{f\PYGZus{}l}\PYG{p}{,} \PYG{n}{x\PYGZus{}n} \PYG{o}{=} \PYG{n}{f\PYGZus{}line}\PYG{p}{(}\PYG{n}{f}\PYG{p}{,} \PYG{n}{x}\PYG{p}{,} \PYG{n}{x\PYGZus{}n}\PYG{p}{)}

        
    \PYG{n}{plt}\PYG{o}{.}\PYG{n}{setp}\PYG{p}{(}\PYG{n}{ax}\PYG{p}{,} \PYG{n}{xticks}\PYG{o}{=}\PYG{p}{[}\PYG{l+m+mi}{0}\PYG{p}{,} \PYG{l+m+mi}{1}\PYG{p}{,} \PYG{l+m+mi}{2}\PYG{p}{]}\PYG{p}{,} \PYG{n}{xticklabels}\PYG{o}{=}\PYG{p}{[}\PYG{l+s+s1}{\PYGZsq{}}\PYG{l+s+s1}{0}\PYG{l+s+s1}{\PYGZsq{}}\PYG{p}{,} \PYG{l+s+s1}{\PYGZsq{}}\PYG{l+s+s1}{1}\PYG{l+s+s1}{\PYGZsq{}}\PYG{p}{,} \PYG{l+s+s1}{\PYGZsq{}}\PYG{l+s+s1}{2}\PYG{l+s+s1}{\PYGZsq{}}\PYG{p}{]}\PYG{p}{,}
             \PYG{n}{yticks}\PYG{o}{=}\PYG{p}{[}\PYG{o}{\PYGZhy{}}\PYG{l+m+mi}{1}\PYG{p}{,} \PYG{l+m+mi}{0}\PYG{p}{,} \PYG{l+m+mi}{1}\PYG{p}{]}\PYG{p}{,} \PYG{n}{yticklabels}\PYG{o}{=}\PYG{p}{[}\PYG{l+s+s1}{\PYGZsq{}}\PYG{l+s+s1}{\PYGZhy{}1}\PYG{l+s+s1}{\PYGZsq{}}\PYG{p}{,} \PYG{l+s+s1}{\PYGZsq{}}\PYG{l+s+s1}{0}\PYG{l+s+s1}{\PYGZsq{}}\PYG{p}{,} \PYG{l+s+s1}{\PYGZsq{}}\PYG{l+s+s1}{\PYGZhy{}1}\PYG{l+s+s1}{\PYGZsq{}}\PYG{p}{]}\PYG{p}{)}
    \PYG{n}{ax}\PYG{o}{.}\PYG{n}{set\PYGZus{}ylim}\PYG{p}{(}\PYG{o}{\PYGZhy{}}\PYG{l+m+mi}{1}\PYG{p}{,} \PYG{l+m+mi}{1}\PYG{p}{)}
    \PYG{n}{ax}\PYG{o}{.}\PYG{n}{set\PYGZus{}xlim}\PYG{p}{(}\PYG{l+m+mi}{0}\PYG{p}{,} \PYG{l+m+mf}{2.2}\PYG{p}{)}
    \PYG{n}{ax}\PYG{o}{.}\PYG{n}{legend}\PYG{p}{(}\PYG{n}{loc}\PYG{o}{=}\PYG{l+m+mi}{4}\PYG{p}{)}
    \PYG{n}{plt}\PYG{o}{.}\PYG{n}{show}\PYG{p}{(}\PYG{p}{)}
\end{sphinxVerbatim}

\end{sphinxuseclass}\end{sphinxVerbatimInput}

\end{sphinxuseclass}
\begin{sphinxuseclass}{cell}\begin{sphinxVerbatimInput}

\begin{sphinxuseclass}{cell_input}
\begin{sphinxVerbatim}[commandchars=\\\{\}]
\PYG{n}{fig}\PYG{p}{,} \PYG{n}{ax} \PYG{o}{=} \PYG{n}{plt}\PYG{o}{.}\PYG{n}{subplots}\PYG{p}{(}\PYG{l+m+mi}{1}\PYG{p}{,} \PYG{n}{figsize}\PYG{o}{=}\PYG{p}{(}\PYG{l+m+mi}{13}\PYG{p}{,} \PYG{l+m+mf}{6.5}\PYG{p}{)}\PYG{p}{)}
\PYG{n}{order} \PYG{o}{=} \PYG{n}{widgets}\PYG{o}{.}\PYG{n}{IntSlider}\PYG{p}{(}\PYG{n+nb}{min}\PYG{o}{=}\PYG{l+m+mi}{1}\PYG{p}{,} \PYG{n+nb}{max}\PYG{o}{=}\PYG{l+m+mi}{6}\PYG{p}{,} \PYG{n}{value}\PYG{o}{=}\PYG{l+m+mi}{1}\PYG{p}{,} \PYG{n}{description}\PYG{o}{=}\PYG{l+s+s1}{\PYGZsq{}}\PYG{l+s+s1}{order}\PYG{l+s+s1}{\PYGZsq{}}\PYG{p}{)}
\PYG{n}{widgets}\PYG{o}{.}\PYG{n}{interactive}\PYG{p}{(}\PYG{n}{update\PYGZus{}plot}\PYG{p}{,} \PYG{n}{order}\PYG{o}{=}\PYG{n}{order}\PYG{p}{)}
\end{sphinxVerbatim}

\end{sphinxuseclass}\end{sphinxVerbatimInput}
\begin{sphinxVerbatimOutput}

\begin{sphinxuseclass}{cell_output}
\begin{sphinxVerbatim}[commandchars=\\\{\}]
\PYGZlt{}IPython.core.display.Javascript object\PYGZgt{}
\end{sphinxVerbatim}

\begin{sphinxVerbatim}[commandchars=\\\{\}]
\PYGZlt{}IPython.core.display.HTML object\PYGZgt{}
\end{sphinxVerbatim}

\begin{sphinxVerbatim}[commandchars=\\\{\}]
interactive(children=(IntSlider(value=1, description=\PYGZsq{}order\PYGZsq{}, max=6, min=1), Output()), \PYGZus{}dom\PYGZus{}classes=(\PYGZsq{}widget\PYGZhy{}…
\end{sphinxVerbatim}

\end{sphinxuseclass}\end{sphinxVerbatimOutput}

\end{sphinxuseclass}

\chapter{Let’s see how this method works}
\label{\detokenize{testNewton-outdatewarning:let-s-see-how-this-method-works}}
\sphinxAtStartPar
let’s use the point\sphinxhyphen{}slope form of a linear equation
\begin{equation*}
\begin{split}
y - y_1 = m(x - x_1)
\end{split}
\end{equation*}
\sphinxAtStartPar
We have a point \(x_1\) and \(f(x_1)=y_1\)

\sphinxAtStartPar
and a slope \(m= f'(x_1)\)

\sphinxAtStartPar
so let’s plug in and find the \(x\) intercept
\begin{equation*}
\begin{split}
y - f(x_1) = f'(x_1)(x - x_1)
\end{split}
\end{equation*}
\sphinxAtStartPar
Set \(y=0\) and solve for \(x_2\)
\begin{equation*}
\begin{split}
 - f(x_1) = f'(x_1)(x_2 - x_1)
\end{split}
\end{equation*}\begin{equation*}
\begin{split}
-\frac{f(x_1)}{f'(x_1)} = x_2 - x_1
\end{split}
\end{equation*}\begin{equation*}
\begin{split}
x_2 = x_1 - \frac{f(x_1)}{f'(x_1)}
\end{split}
\end{equation*}
\sphinxAtStartPar
or in general terms
\begin{equation*}
\begin{split}
x_{n+1} = x_n - \frac{f(x_n)}{f'(x_n)}
\end{split}
\end{equation*}

\chapter{Let’s look at how to code this}
\label{\detokenize{testNewton-outdatewarning:let-s-look-at-how-to-code-this}}
\begin{sphinxuseclass}{cell}\begin{sphinxVerbatimInput}

\begin{sphinxuseclass}{cell_input}
\begin{sphinxVerbatim}[commandchars=\\\{\}]
\PYG{k+kn}{import} \PYG{n+nn}{numpy} \PYG{k}{as} \PYG{n+nn}{np}
\PYG{k+kn}{from} \PYG{n+nn}{scipy}\PYG{n+nn}{.}\PYG{n+nn}{misc} \PYG{k+kn}{import} \PYG{n}{derivative}

\PYG{n}{x\PYGZus{}n} \PYG{o}{=} \PYG{l+m+mi}{2}
\PYG{n}{x} \PYG{o}{=} \PYG{n}{np}\PYG{o}{.}\PYG{n}{linspace}\PYG{p}{(}\PYG{l+m+mf}{0.2}\PYG{p}{,} \PYG{l+m+mf}{2.2}\PYG{p}{,} \PYG{l+m+mi}{500}\PYG{p}{)}

\PYG{k}{def} \PYG{n+nf}{f}\PYG{p}{(}\PYG{n}{x}\PYG{p}{)}\PYG{p}{:}
    \PYG{k}{return} \PYG{n}{np}\PYG{o}{.}\PYG{n}{log}\PYG{p}{(}\PYG{n}{x}\PYG{p}{)}

\PYG{k}{def} \PYG{n+nf}{x\PYGZus{}next}\PYG{p}{(}\PYG{n}{f}\PYG{p}{,} \PYG{n}{x}\PYG{p}{,} \PYG{n}{x\PYGZus{}n}\PYG{p}{)}\PYG{p}{:}
    \PYG{n}{slope} \PYG{o}{=} \PYG{n}{derivative}\PYG{p}{(}\PYG{n}{f}\PYG{p}{,} \PYG{n}{x\PYGZus{}n}\PYG{p}{,} \PYG{n}{dx}\PYG{o}{=}\PYG{l+m+mf}{0.1}\PYG{p}{)}
    \PYG{k}{return} \PYG{n}{x\PYGZus{}n} \PYG{o}{\PYGZhy{}} \PYG{n}{f}\PYG{p}{(}\PYG{n}{x\PYGZus{}n}\PYG{p}{)} \PYG{o}{/} \PYG{n}{slope}

\PYG{k}{for} \PYG{n}{n} \PYG{o+ow}{in} \PYG{n+nb}{range}\PYG{p}{(}\PYG{l+m+mi}{6}\PYG{p}{)}\PYG{p}{:}
    \PYG{n+nb}{print}\PYG{p}{(}\PYG{l+s+s1}{\PYGZsq{}}\PYG{l+s+s1}{x\PYGZus{}}\PYG{l+s+si}{\PYGZob{}\PYGZcb{}}\PYG{l+s+s1}{ = }\PYG{l+s+si}{\PYGZob{}:.6f\PYGZcb{}}\PYG{l+s+s1}{\PYGZsq{}}\PYG{o}{.}\PYG{n}{format}\PYG{p}{(}\PYG{n}{n} \PYG{o}{+} \PYG{l+m+mi}{1}\PYG{p}{,} \PYG{n}{x\PYGZus{}n}\PYG{p}{)}\PYG{p}{)}
    \PYG{n}{x\PYGZus{}n} \PYG{o}{=} \PYG{n}{x\PYGZus{}next}\PYG{p}{(}\PYG{n}{f}\PYG{p}{,} \PYG{n}{x}\PYG{p}{,} \PYG{n}{x\PYGZus{}n}\PYG{p}{)}
\end{sphinxVerbatim}

\end{sphinxuseclass}\end{sphinxVerbatimInput}
\begin{sphinxVerbatimOutput}

\begin{sphinxuseclass}{cell_output}
\begin{sphinxVerbatim}[commandchars=\\\{\}]
x\PYGZus{}1 = 2.000000
x\PYGZus{}2 = 0.614862
x\PYGZus{}3 = 0.911249
x\PYGZus{}4 = 0.995599
x\PYGZus{}5 = 0.999975
x\PYGZus{}6 = 1.000000
\end{sphinxVerbatim}

\begin{sphinxVerbatim}[commandchars=\\\{\}]
/var/folders/33/krstvgns2rncl74r18tkv6\PYGZus{}80000gn/T/ipykernel\PYGZus{}32058/3440021237.py:11: DeprecationWarning: scipy.misc.derivative is deprecated in SciPy v1.10.0; and will be completely removed in SciPy v1.12.0. You may consider using findiff: https://github.com/maroba/findiff or numdifftools: https://github.com/pbrod/numdifftools
  slope = derivative(f, x\PYGZus{}n, dx=0.1)
\end{sphinxVerbatim}

\end{sphinxuseclass}\end{sphinxVerbatimOutput}

\end{sphinxuseclass}
\begin{sphinxthebibliography}{HdHPK14}
\bibitem[HdHPK14]{markdown:id3}
\sphinxAtStartPar
Christopher Ramsay Holdgraf, Wendy de Heer, Brian N. Pasley, and Robert T. Knight. Evidence for Predictive Coding in Human Auditory Cortex. In \sphinxstyleemphasis{International Conference on Cognitive Neuroscience}. Brisbane, Australia, Australia, 2014. Frontiers in Neuroscience.
\end{sphinxthebibliography}







\renewcommand{\indexname}{Index}
\printindex
\end{document}